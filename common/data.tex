%%% Основные сведения %%%
\newcommand{\thesisAuthorLastName}{{Дмитриев}}
\newcommand{\thesisAuthorOtherNames}{{Алексей Юрьевич}}
\newcommand{\thesisAuthorInitials}{{А.\,Ю.}}
\newcommand{\thesisAuthor}             % Диссертация, ФИО автора
{%
    \texorpdfstring{% \texorpdfstring takes two arguments and uses the first for (La)TeX and the second for pdf
        \thesisAuthorLastName~\thesisAuthorOtherNames% так будет отображаться на титульном листе или в тексте, где будет использоваться переменная
    }{%
        \thesisAuthorLastName, \thesisAuthorOtherNames% эта запись для свойств pdf-файла. В таком виде, если pdf будет обработан программами для сбора библиографических сведений, будет правильно представлена фамилия.
    }
}
\newcommand{\thesisAuthorShort}        % Диссертация, ФИО автора инициалами
{\thesisAuthorInitials~\thesisAuthorLastName}
%\newcommand{\thesisUdk}                % Диссертация, УДК
%{{xxx.xxx}}
\newcommand{\thesisTitle}              % Диссертация, название
{{Исследование нелинейных квантово-оптических эффектов при рассеянии света на сверхпроводниковом кубите в одномерном пространстве}}
\newcommand{\thesisSpecialtyNumber}    % Диссертация, специальность, номер
{{01.04.07}}
\newcommand{\thesisSpecialtyTitle}     % Диссертация, специальность, название
{{Физика конденсированного состояния}}
\newcommand{\thesisDegree}             % Диссертация, ученая степень
{{кандидата физико-математических наук}}
\newcommand{\thesisDegreeShort}        % Диссертация, ученая степень, краткая запись
{{к.ф-м.н}}
\newcommand{\thesisCity}               % Диссертация, город написания диссертации
{{Долгопрудный}}
\newcommand{\thesisYear}               % Диссертация, год написания диссертации
{{2021}}
\newcommand{\thesisOrganization}       % Диссертация, организация
{{Федеральное государственное автономное образовательное учреждение высшего образования <<Московский физико-технический институт (национальный исследовательский университет)>>}}
\newcommand{\thesisOrganizationShort}  % Диссертация, краткое название организации для доклада
{{МФТИ}}

\newcommand{\thesisInOrganization}     % Диссертация, организация в предложном падеже: Работа выполнена в ...
{{Федеральном государственном автономном образовательном учреждении высшего образования «Московский физико-технический институт (государственный университет)»}}

\newcommand{\supervisorFio}            % Научный руководитель, ФИО
{{Астафьев Олег Владимирович}}
\newcommand{\supervisorRegalia}        % Научный руководитель, регалии
{{кандидат физико-математических наук,~профессор}}
\newcommand{\supervisorFioShort}       % Научный руководитель, ФИО
{{О.\,В.~Астафьев}}
\newcommand{\supervisorRegaliaShort}   % Научный руководитель, регалии
{{к.ф-м.н..,~проф.}}


\newcommand{\opponentOneFio}           % Оппонент 1, ФИО
{{Устинов Алексей Викторович}}
\newcommand{\opponentOneRegalia}       % Оппонент 1, регалии
{{доктор~физ.-мат.~наук}}
\newcommand{\opponentOneJobPlace}      % Оппонент 1, место работы
{{Московский институт стали и сплавов (МИСиС)}}
\newcommand{\opponentOneJobPost}       % Оппонент 1, должность
{{профессор, заведующий лабораторией сверхпроводящих метаматериалов}}

\newcommand{\opponentTwoFio}           % Оппонент 2, ФИО
{{Махлин Юрий Генрихович}}
\newcommand{\opponentTwoRegalia}       % Оппонент 2, регалии
{{Член-корр.~РАН, доктор физ.-мат. наук}}
\newcommand{\opponentTwoJobPlace}      % Оппонент 2, место работы
{{ИТФ им. Ландау}}
\newcommand{\opponentTwoJobPost}       % Оппонент 2, должность
{{Ведущий научный сотрудник}}

\newcommand{\leadingOrganizationTitle} % Ведущая организация, дополнительные строки
{{Федеральное государственное бюджетное образовательное учреждение высшего образования «Московский государственный университет имени М.В.Ломоносова»}}

\newcommand{\defenseDate}              % Защита, дата
{{15~августа~2021~г.~в~15~часов~00~минут}}
\newcommand{\defenseCouncilNumber}     % Защита, номер диссертационного совета
{{ЛФИ\,01.04.07.008}}
\newcommand{\defenseCouncilTitle}      % Защита, учреждение диссертационного совета
{{Московском~Физико-Техническом~Институте}}
\newcommand{\defenseCouncilAddress}    % Защита, адрес учреждение диссертационного совета
{{МО,~г.Долгопрудный, Институтский пер., 9}}
\newcommand{\defenseCouncilPhone}      % Телефон для справок
{{+7~(916)~400-73-42}}

\newcommand{\defenseSecretaryFio}      % Секретарь диссертационного совета, ФИО
{{Останина Валентина Олеговна}}
\newcommand{\defenseSecretaryRegalia}  % Секретарь диссертационного совета, регалии
{{к.ф-м.н.}}            % Для сокращений есть ГОСТы, например: ГОСТ Р 7.0.12-2011 + http://base.garant.ru/179724/#block_30000

\newcommand{\synopsisLibrary}          % Автореферат, название библиотеки
{{Московского~Физико-Технического~Института~и~по~адресу:~\\https://mipt.ru/education/post-graduate/soiskateli-fiziko-matematicheskie-nauki.php}}
\newcommand{\synopsisDate}             % Автореферат, дата рассылки
{{01 июня 2021 года}}

% To avoid conflict with beamer class use \providecommand
\providecommand{\keywords}%            % Ключевые слова для метаданных PDF диссертации и автореферата
{}
