{\actuality} 
Макроскопические сверхпроводящие квантовые цепи (СКЦ) --- одна из наиболее активно развивающихся областей современной экспериментальной квантовой физики. Временем её непосредственного зарождения можно считать  1999"--~2001 гг., когда в нескольких пионерских работах была показана  возможность создания макроскопических когерентных квантовых объектов на основе сверхпроводников и возможность приготовления и контроля квантовых состояний таких систем. Интерес к таким системам вырос чрезвычайно быстро: ученые поняли, что на основе СКЦ принципиально возможно построить устройства, выполняющие квантовые операции, а значит, создать квантовый процессор. За 15 с небольшим лет мировым физическим сообществом проделана огромная как теоретическая, так и экспериментальная работа для реализации квантовых вычислений и демонстрации квантовых алгоритмов при помощи устройств на базе СКЦ. Поэтому в дальнейшем будем употреблять для таких схем термин <<кубит>>, подразумевая квантовую цепь, хотя и не обязательно обладающую лишь двумя квантовыми уровнями (Там, где это может привести к путанице, будут даваться соотвествующие разъяснения).  Кратко перечислим полученные мировым научным сообществом результаты:
\begin{enumerate}
	\item Разработан универсальный формализм квантования произвольного кубита, позволяющий как рассчитать энергетический спектр квантовой системы, так и учесть эффекты внешнего воздействия на нее.
	\item Разработан теоретический подход, \cite{Sychev}описывающий взаиможействие кубита на чипе и квантованной моды поля, например, в копланарном резонаторе, расположенном на том же чипе и связанном с кубитом через общие электрические элементы --- емкости и индуктивности. Поскольку многое было заимствовано из т.н. квантовой электродинамики резонаторных полостей (англ. \textit{cavity-QED}), то данная теоретическая модель по аналогии носит название квантовой электродинамики цепей (англ. \textit{circuit-QED}, или cQED). Она описывает все основные эффекты взаимодействия микроволнового поля и кубита, играющего роль <<\textit{искусственного атома}>>.
	\item Развиты методики изготовления кубитов, приготовления и считывания квантовых состояний. Экспериментально изучены несколько различных типов сверхпроводниковых кубитов, и определены схемы, наиболее перспективные для достижения больших времен релаксации и дефазировки состояний кубита. Времена $T_1$ и $T_2$ выросли на 5 порядков: от $10^{-8}$~с для первых образцов до $10^{-4}\div10^{-3}$~с для современных образцов. Более того, в последние 5-7 лет сформировалась интересная тенденция, получившая имя \textit{закон Шёлкопфа}, по аналогии с известным законом Мура в кремниевой электронике: максимальные достижимые времена $T_1$ и $T_2$ кубитов растут с течением времени по показательному закону. Детально изучены основные факторы, приводящие к релаксации и дефазировке в сверхпроводниковых кубитах, а именно: двухуровневые системы и свободные спины в подложке, квазичастицы, качество поверхности <<металл-подложка>>.
	\item Реализованы всевозможные типы одно- и двухкубитных операций, изобретены и реализованы различные экспериментальные техники для оптимизации качества гейтов. В результате, у ведущих научных групп ошибки в среднем не превышают 1\% для однокубитных гейтов, 2--5\% для двухкубитных гейтов (в зависимости от реализации гейтов, типа и числа кубитов на чипе)
	\item Разработаны и реализованы нелинейные параметрические усилители на основе туннельных контактов с квантовым уровнем шума, что позволяет проводить единовременное считывание (англ. \textit{single-shot readout}) состояния кубита (проецирующее $\sigma_z$-измерение) с точностью более 95\%.
\end{enumerate}
Данный список можно продолжать и далее, но остановимся на главном. Перечисленные успехи позволяют рассуждать о возможном создании полномасштабных универсальных квантовых вычислительных устройств на базе СКЦ, демонстрирующих квантовое превосходство (англ. \textit{quantum supremacy}), и именно к этой цели в настоящий момент направлены проекты больших исследовательских групп при корпорациях \textit{Google}, \textit{IBM} и \textit{Intel}, а также ряд других проектов некоторых частных компаний (напр.,~\textit{Rigetti Inc.}). Эта деятельность сопровождается огромным количеством интересных научных результатов в области фундаментальной сверхпроводимости, квантовой электродинамики цепей, и даже образованием новых научных областей, как например, физика квантовых сверхпроводящих метаматериалов, фотоника в микроволновом диапазоне и нелинейная квантовая оптика, где в качестве среды выступают одиночные кубиты или небольшие массивы кубитов. Последние две области особенно интересны, так как используя контроль состояний одиночных искеусственных атомов, можно изучать очень интересные режимы генерации, поглощения и рассеяния света, которые труднодоступны как при изучении света в оптическом диапазоне, взаимодействующего с <<природными>> атомами, так и при использовании ридберговских состояний атомов, в которых они обладают большим дипольным моментом и хорошо взаимодействуют с микроволновым излучением.  В частности, результаты, полученные автором и описываемые в рамках данной диссертации, относятся именно к области нелинейной микроволновой квантовой оптики. Суммируя все вышеперечисленное, можно сделать вывод о значительной актуальности научных исследований в области сверхпроводниковых квантовых систем, и в частности, тех работ, о которых пойдет речь в данной диссертации.



%Обзор, введение в тему, обозначение места данной работы в
%мировых исследованиях и~т.\:п., можно использовать ссылки на~другие
%работы\ifnumequal{\value{bibliosel}}{1}{~\autocite{Gosele1999161}}{}
%(если их~нет, то~в~автореферате
%автоматически пропадёт раздел <<Список литературы>>). Внимание! Ссылки
%на~другие работы в разделе общей характеристики работы можно
%использовать только при использовании \verb!biblatex! (из-за технических
%ограничений \verb!bibtex8!. Это связано с тем, что одна
%и~та~же~характеристика используются и~в~тексте диссертации, и в
%автореферате. В~последнем, согласно ГОСТ, должен присутствовать список
%работ автора по~теме диссертации, а~\verb!bibtex8! не~умеет выводить в одном
%файле два списка литературы).
%При использовании \verb!biblatex! возможно использование исключительно
%в~автореферате подстрочных ссылок
%для других работ командой \verb!\autocite!, а~также цитирование
%собственных работ командой \verb!\cite!. Для этого в~файле
%\verb!Synopsis/setup.tex! необходимо присвоить положительное значение
%счётчику \verb!\setcounter{usefootcite}{1}!.
%
%Для генерации содержимого титульного листа автореферата, диссертации
%и~презентации используются данные из файла \verb!common/data.tex!. Если,
%например, вы меняете название диссертации, то оно автоматически
%появится в~итоговых файлах после очередного запуска \LaTeX. Согласно
%ГОСТ 7.0.11-2011 <<5.1.1 Титульный лист является первой страницей
%диссертации, служит источником информации, необходимой для обработки и
%поиска документа>>. Наличие логотипа организации на титульном листе
%упрощает обработку и поиск, для этого разметите логотип вашей
%организации в папке images в формате PDF (лучше найти его в векторном
%варианте, чтобы он хорошо смотрелся при печати) под именем
%\verb!logo.pdf!. Настроить размер изображения с логотипом можно
%в~соответствующих местах файлов \verb!title.tex!  отдельно для
%диссертации и автореферата. Если вам логотип не~нужен, то просто
%удалите файл с логотипом.
%
%\ifsynopsis
%Этот абзац появляется только в~автореферате.
%Для формирования блоков, которые будут обрабатываться только в~автореферате,
%заведена проверка условия \verb!\!\verb!ifsynopsis!.
%Значение условия задаётся в~основном файле документа (\verb!synopsis.tex! для
%автореферата).
%\else
%Этот абзац появляется только в~диссертации.
%Через проверку условия \verb!\!\verb!ifsynopsis!, задаваемого в~основном файле
%документа (\verb!dissertation.tex! для диссертации), можно сделать новую
%команду, обеспечивающую появление цитаты в~диссертации, но~не~в~автореферате.
%\fi

% {\progress} 
% Этот раздел должен быть отдельным структурным элементом по
% ГОСТ, но он, как правило, включается в описание актуальности
% темы. Нужен он отдельным структурынм элемементом или нет ---
% смотрите другие диссертации вашего совета, скорее всего не нужен.

{\aim} данной работы является экспериментальное изучение процессов трех- и четырехволнового смешения распространяющегося микроволнового света на одиночном искусственном атоме, сильно связанном с внешним пространством, а также поиск новых интересных особенностей этих процессов, обусловленных присутствием квантового объекта в качестве рассеивателя.

Для~достижения поставленной цели необходимо было решить следующие {\tasks}:
\begin{enumerate}
	\item Теоретический расчет, проектирование и изготовление образцов одиночных сверхпроводниковых кубитов, связанных с открытым полупространством --- копланарной линией на чипе.
	\item Разработка и сборка различных типов экспериментальных схем на микроволновых компонентах, необходимых для измерения кубитов в криостате растворения и обеспечивающих правильную работу кубитов.
	\item Проведение измерений спектров кубитов, измерение параметров связи, времен релаксации и дефазировки с использованием импульсных техник измерений.
	\item Выработка концепции и реализация эксперимента по рассеянию микроволн нескольких частот на двухуровневой системе, наблюдение компонент четырехволнового смешения на кубите. Анализ и описание полученных результатов.
	\item Исследование спектра когерентного излучения, рассеянного цикличной трехуровневой системой (со схемой уровней типа $\Delta$). Наблюдение трехволнового смешивания, его теоретическое описание.
	
\end{enumerate}

{\methods} При изготовлении образцов использовались стандартные процессы нанофабрикации. Для изготовления структур с размерами от 2 мкм использовалась лазерная литография, для изготовления кубитов с размерами менее 200 нм --- электронная литография. Кубит формировался методом двухуглового теневого напыления через предварительно проявленную маску. Сборка измерительных схем проводилась в соответствии с общепринятыми принципами низкотемпературных микроволновых измерений, позволяющими изолировать структуры от теплового шума и усиливать рассеянный сигнал, по мощности близкий к однофотонному. Измерения проводились при помощи векторного анализатора цепей и спектрального анализатора. 
Проведение измерений автоматизировалось при помощи драйверов и измерительных скриптов, разработанных при помощи высокоуровневого языка Python в среде Jupyter Notebook и позволяющих управлять приборами, снимать, обрабатывать и визуализировать экспериментальные данные. Использовались библиотеки Pyvisa, Numpy, Matplotlib и др.

{\novelty}
\begin{enumerate}
  \item Впервые продемонстрирован эффект четырёхволнового смешивания при рассеянии двух резонансных мод на одиночном потоковом кубите, сильно связанном с континуумом электромагнитных мод в копланарной линии. Показано наличие побочных спектральных компонент в составе когерентного излучения, рассеянного кубитом. 
  \item Получена аналитическая формула для расчета спектральной интенсивности боковых компонент, возникающих при смешивании волн произвольного порядка. Результаты расчетов хорошо согласуются с экспериментальными данными. 
  \item Впервые изучен процесс смешивания двух коротких микроволновых импульсов на кубите и продемонстрированы  появление Бесселевских Раби-осцилляций (см.~ниже).
  \item Впервые продемонстрировано смешивание квантового состояния поля в первой из мод, образующегося за счет излучения кубита из предварительно приготовленного состояния суперпозиции, и классического состояния поля во второй из мод, сформированного электромагнитным импульсом - т.н. \textit{квантовое смешивание волн}.
  \item Впервые показано трехволновое смешивание при рассеянии резонансных сигналов на одиночном трехуровневом искусственном атоме, уровни которого образуют $\Delta$-систему.
\end{enumerate}

%{\influence} 

{\defpositions}
\begin{enumerate}
  \item При облучении кубита частоты $\omega_0$, сильно связанного с одномерным пространством, двумя непрерывными сигналами частот $\omega_+$=$\omega_0$+$\delta$~и~$\omega_-$=$\omega_0$--$\delta$, находящимися в резонансе с кубитом ($\omega_-$,~$\omega_-\ll\Gamma_1$), в спектре когерентно рассеянного излучения возникают <<боковые>> (по аналогии с англ. \textit{sideband}, далее без кавычек) компоненты с частотами $\omega_{\pm(2k+1)}=\pm(k+1)\omega_{\pm}\mp k\omega_{\mp}$, где $k$ -- целое положительное число.
  \item Появление боковых компонент и их спектральную интенсивность компонент можно объяснить процессами нелинейного смешивания первоначальных сигналов при рассеянии на кубите, играющем роль нелинейной оптической среды. Также этот эффект можно интерпретировать в терминах многофотонного рассеяния с участием $2k+2$ фотонов.
  \item При облучении кубита двумя короткими импульсами с частотами $\omega_+$ и $\omega_-$, амплитуды которых одинаковы и равны $\Omega$, а длительности $t$ значительно меньше чем $T_1,T_2$ кубита, временная динамика системы вкупе с эффектом нелинейного смешивания приводят к появлению Бесселевских Раби-осцилляции в боковых частотных компонентах: спектральная интенсивность компоненты с частотой $\omega_{\pm(2k+1)}$ имеет зависимость вида $I \propto J^2_{2k+1}(2\Omega t)$, где $J$ -- функция Бесселя 1-го рода. 
  \item При введении задержки импульсов с частотой $\omega_-$ относительно импульсов с частотой $\omega_+$  характер спектра кардинально меняется: вместо большого числа боковых компонент возникает лишь одна из них: $\omega_{-3} = 2\omega_- - \omega_+$. Это обусловлено фотонной статистикой состояний света в моде $\omega_+$: из-за переизлучения света двухуровневой системой в этом состоянии не может быть более 1 фотона, и нелинейные процессы высшего порядка оказываются запрещенными. Похожая картина возникает при рассеянии света на трехуровневой системе, так как состояние с 2-мя фотонами <<разрешает>> большее количество многофотонных процессов.
  \item При рассеянии двух резонансных микроволновых сигналов на трёхуровневой $\Delta$-системе возникает трехволновое смешивание. Интенсивность третьей компоненты, появляющейся за счет смешивания, описывается решением Блоховских уравнений для данной системы. 
\end{enumerate}
%В папке Documents можно ознакомиться в решением совета из Томского ГУ
%в~файле \verb+Def_positions.pdf+, где обоснованно даются рекомендации
%по~формулировкам защищаемых положений. 

{\reliability} полученных результатов обеспечивается соответствием между аналитическими вычислениями и экспериментальными данными. Данное соответствие имеет место по всем положениям, выносимым на защиту. 

{\probation}
Основные результаты работы представлялись на различных международных конференциях, семинарах и воркшопах, например: Workshop on Physics and Applications of Superconductivity, Кембриджский Университет, Великобритания; Quantum Simulation and Computation Summer School , Гётеборг, Швеция;  Мезоскопические структуры в  фундаментальных и прикладных исследованиях, Новосибирск, Россия; Superconducting Hybrid Nanostructures: Physics and Applications, Долгопрудный, Россия; Quantum Coherent Phenomena at Nanoscale, Петровац, Черногория; Superconductor-based sensors and quantum technologies, Москва, Россия; 2nd International Conference on Quantum Physics and Quantum Technology, Берлин, Германия; 4th International conference on quantum technologies, Москва, Россия; 20th International Seminar <<Superconducting Quantum Circuits>>, Ишгль, Австрия; 1-я всероссийская школа по квантовым технологиям, Сочи, Россия, и~др. Результаты также неоднократно докладывались и обсуждались на семинарах Лаборатории искусственных квантовых систем МФТИ


{\contribution} Автор принимал активное участие в постановке задач, фабрикации образцов, проведении экспериментов, обработке данных и интерпретации результатов. Все заявленные результаты получены при непосредственном участии автора диссертации.

%\publications\ Основные результаты по теме диссертации изложены в ХХ печатных изданиях~\cite{Sokolov,Gaidaenko,Lermontov,Management},
%Х из которых изданы в журналах, рекомендованных ВАК~\cite{Sokolov,Gaidaenko}, 
%ХХ --- в тезисах докладов~\cite{Lermontov,Management}.

\ifnumequal{\value{bibliosel}}{0}{% Встроенная реализация с загрузкой файла через движок bibtex8
    \publications\ Материалы диссертации изложены в 5 публикациях, 3 из которых опубликованы в печатных изданиях, из них
    2 --- в журналах, индексируемых в Web of Science, 1 публикация  --- в сборниках трудов и тезисов конференций. Также 2 публикации размещены на архиве препринтов arXiv.org и находятся в процессе рецензирования в международных научных изданиях. %
}{% Реализация пакетом biblatex через движок biber
%Сделана отдельная секция, чтобы не отображались в списке цитированных материалов
    \begin{refsection}[vak,papers,conf]% Подсчет и нумерация авторских работ. Засчитываются только те, которые были прописаны внутри \nocite{}.
        %Чтобы сменить порядок разделов в сгрупированном списке литературы необходимо перетасовать следующие три строчки, а также команды в разделе \newcommand*{\insertbiblioauthorgrouped} в файле biblio/biblatex.tex
        \printbibliography[heading=countauthorvak, env=countauthorvak, keyword=biblioauthorvak, section=1]%
        \printbibliography[heading=countauthorconf, env=countauthorconf, keyword=biblioauthorconf, section=1]%
        \printbibliography[heading=countauthornotvak, env=countauthornotvak, keyword=biblioauthornotvak, section=1]%
        \printbibliography[heading=countauthor, env=countauthor, keyword=biblioauthor, section=1]%
        \nocite{%Порядок перечисления в этом блоке определяет порядок вывода в списке публикаций автора
                vakbib1,vakbib2,%
                confbib1,confbib2,%
                bib1,bib2,%
        }%
        \publications\ Основные результаты по теме диссертации изложены в~\arabic{citeauthor}~печатных изданиях, 
        \arabic{citeauthorvak} из которых изданы в журналах, индексируемых в системе Web of Science , 
        \arabic{citeauthorconf} "--- в~тезисах докладов.
    \end{refsection}
    \begin{refsection}[vak,papers,conf]%Блок, позволяющий отобрать из всех работ автора наиболее значимые, и только их вывести в автореферате, но считать в блоке выше общее число работ
        \printbibliography[heading=countauthorvak, env=countauthorvak, keyword=biblioauthorvak, section=2]%
        \printbibliography[heading=countauthornotvak, env=countauthornotvak, keyword=biblioauthornotvak, section=2]%
        \printbibliography[heading=countauthorconf, env=countauthorconf, keyword=biblioauthorconf, section=2]%
        \printbibliography[heading=countauthor, env=countauthor, keyword=biblioauthor, section=2]%
        \nocite{vakbib2}%vak
        \nocite{bib1}%notvak
        \nocite{confbib1}%conf
    \end{refsection}
}
%При использовании пакета \verb!biblatex! для автоматического подсчёта
%количества публикаций автора по теме диссертации, необходимо
%их~здесь перечислить с использованием команды \verb!\nocite!.
