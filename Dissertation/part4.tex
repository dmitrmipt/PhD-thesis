\chapter{Двухчастотное волновое смешение в импульсном режиме}\label{ch: q_mixing}

В данной главе будут описаны эффекты волнового смешения, возникающие при импульсной бихроматической накачке двухуровневой системы. В этих экспериментах на кубит подаются последовательности прямоугольных или близких к ним по форме импульсов. Несущие частоты этих последовательностей незначительно отстроены от резонанса кубита, длительности импульсов варьируются от $0$ до нескольких $T_1$ кубита, а промежутки между импульсами значительно превышают $T_1$. Качественная картина спектра эластичного рассеяния претерпевает значительные изменения по сравнению с непрерывной накачкой. Например, при рассеянии синхронизированных прямоугольных импульсов на частотах $\omega_{pm}$ наблюдается бесселевская динамика боковых компонент сигнала в зависимости от эффективной длительности импульса $\Omega\Delta t$. Совершенно новый физический эффект возникает при введении задержки между последовательностями импульсов. Длительность задержки должна превосходить длительность импульса для того, чтобы отдельные импульсы на частоте $\omega_+$ попадали на кубит раньше импульсов на частоте $\omega_-$ и не перекрывались с ними. При этом кардинально меняется вид спектра эластичного рассеяния. Вместо большого количества симметричных пиков мы наблюдаем лишь один дополнительный пик на частоте $2\omega_--\omega_+$. Этому эффекту можно дать простое качественное объяснение, напрямую вытекающее из свойства кубита поглощать не более чем единичный квант поля. В дополнение к этому, мы изучаем рассеяние последовательностей импульсов на трехуровневой эквидистантной системе, которая возникает при некотором значении внешнего магнитного потока, проходящего через петлю изучаемого нами потокового кубита. Помимо к двух компонент на исходных несущих частотах и одной компоненте от четырехволнового смешения, к эластичному спектру добавляется еще две компоненты, соотвествующие двухфотонным процессам, что подтверждает справедливость качественной интерпретации экспериментальных результатов. Последние два режима мы будем называть \textit{квантовым волновым смешением}, поскольку оно обладает рядом необычных свойств, обусловленных квантовостью сверхпроводникового искусственного атома. В дополнение к экспериментальным данным, в данном разделе проведен численный расчет спектра, согласующийся с экспериментом, а также дано строгое теоретическое обоснование трехпикового спектра рассеяния на двухуровневой системе и пятипикового спектра рассеяния на трехуровневой системе на основе формализма вторичного квантования.  
\section{Случай синхронных импульсов: бесселевская динамика}
Для изучения импульсного смешения при помощи экспериментальной схемы \ref{fig: pulse_setup_1} необходимо запрограммировать генератор импульсов произвольной формы (AWG) так, чтобы на его выходе получить последовательность прямоугольных импульсов. Несущая частота этих импульсов $\omega_{\text{IF}}/2\pi$ может варьироваться от 0 до 100 МГц, а период импульсов $T\gg 1/\Gamma_1$ обеспечивает релаксацию кубита после импульса за время, предшествующее появлению следующего импульса. Длительности импульсов $\delta t$ также меняются произвольно, технические возможности AWG позволяют получить импульсы длительностью от 2 нс и выше. Импульсный сигнал с выхода AWG попадает на квадратурный смеситель, где смешивается с сигналом локального осциллятора, приобретая таким образом несущую частоту $\omega_{d} = \omega_{\text{LO}}\pm\omega_{\text{IF}}$, где выбор знака произволен и зависит от калибровки квадратурного смесителя. Затем сигнал направляется в криостат, попадает в волновод с искусственным атомом и рассеивается на нем.  Детектирование рассеянного сигнала осуществляется при помощи спектрального анализатора либо при помощи высокоскоростного АЦП. Во втором случае на выходе сигнал испытывает обратное преобразование частоты вниз при помощи еще одного квадратурного смесителя, после чего сигнал на частоте $\omega_{\text{IF}}$ оцифровывается, а квадратуры комплексного сигнала $I$ и $Q$ вычисляются при помощи цифрового преобразования Фурье \cite{sank2014fast}.

Сгенерировав две синхронные последовательности импульсов одинаковой длительности $\Delta t_-\!=\!\Delta t_+\!=\!\Delta t$ на частотах $\omega_+$ и $\omega_-$ описанным способом и подав их на кубит, мы измеряем спектр эластичного рассеяния, усредненный по большому количествую периодов. Это достигается установлением параметра RBW (Resolution Bandwidth) спектрального анализатора до 1 кГц и ниже, что делает время усреднения более 1 мс, тогда как период следования импульсов составляет 1-10 мкс. Очевидно, что длительности $\Delta t$ импульсов оказывают ключевое влияние на характер динамики кубита, и соответственно, определяют свойства рассеянного излучения, поэтому мы снимаем зависимости интенсивностей боковых компонент от эффективной длительности импульсов $\Omega\Delta t$.
Результат этих измерений изображен на Рис \ref{fig: Bessels}. 
\begin{figure}\label{fig: Bessels}
\centering
\includegraphics[width=1\textwidth]{Bessels.pdf}
\caption[Зависимость интенсивности боковых компонент от длительности импульсов $\Delta t$ бихроматической накачки]{Зависимость интенсивности боковых компонент от длительности импульсов бихроматической накачки. (a) Последовательность управляющих импульсов длительностью $\Delta t=2$~нс с периодом $T_r=1$~мкс и часть спектра эластичного рассеяния c когерентыми пиками. (б) Измеренные зависимости потока фотонов в пике от угла поворота $\zeta_{2p+1}(\Omega \Delta t). $. Точками показаны результаты измерений, сплошные линии соответствуют выражениям $J^2_{2p+1}(2\Omega\Delta t)/4$, описывающим эксперимент без подгоночных параметров.}
\end{figure} 
Этот результат достаточно просто объясняется динамикой кубита под воздействием классического поля. Для динамики кубита под действием двух классических импульсов можно получить следующее выражение:
\begin{equation}
	\braket{\sigma_-} = -\frac{1}{2}\sin(2\Omega\Delta t \cos\delta\omega t), 
\end{equation}
которое раскладывается в ряд по бесселевским функциям согласно соотношению Ангера-Якоби:
\begin{equation}
	\langle \sigma^- \rangle = -\sum_{k=-\infty}^\infty (-1)^k J_{2k+1}(2\Omega \Delta t) \cos[(2k+1)\delta\omega t],
\end{equation}
и усреднение спектральных компонент $\langle \sigma^-_{2k+1}\rangle$ с определенным набегом фазы $(2k+1)\delta \omega t$ дает:
\begin{equation}
	\langle \sigma_{2k+1}^-\rangle = \frac{(-1)^k}{2}J_{2k+1}(2\Omega t),
	\label{sclass_spectr} 
\end{equation} 
что соответствует экспериментально представленным зависимостям на Рис.~\ref{fig: Bessels}. Данный результат можно представить как разложение осцилляций Раби, которые наблюдались бы при $\delta\omega =0$, по разным спектральным компонентам.

Вывод результата \eqref{sclass_spectr} достаточно прост и не требует квантовомеханического описания поля, но тем не менее, он не иллюстрирует сущность наблюдаемого волнового смешения, в частности, не выделяет индивидуальный вклад различных многофотонных процессов в излучение и связь с фотонной статистикой излучаемого света. Более того, полуклассическая картина не выявляет физического происхождения ограниченного количества спектральных компонент, которые мы будем наблюдать для случая квантового волнового смешения двух последовательностей импульсов бихроматической накачки с задержкой. Поэтому мы получим бесселевскую динамику также при помощи рассмотрения эволюции кубита под воздействием двух квантованных мод поля, описывающихся операторами $a_\pm$ и $a_\pm^\dag$, взаимодействующих с кубитом. Как мы увидим далее, этот формализм позволяет описать не только случай синхронных импульсов, характеризующийся рассмотренной выше бесселевской динамикой, но и случай последовательных неперекрывающихся во времени импульсов, где спектр кардинально отличается от рассмотренного выше режима. Для начала, выпишем и обсудим несколько известных результатов, касающихся взаимодействия для одной моды внутри волновода и кубита, связанного с этим волноводомё. 

Будем рассматривать двухуровневый атом с основным и возбужденными состояниями $\ket{g}$ и $\ket{e}$ и энергией $\hbar \omega_0$, взаимодействующий с квантованным полем на частоте $\omega$. Эта система может характеризоваться гамильтонианом, записанным в представлении взаимодействия:
\begin{equation}
	H = i\hbar g (a^\dag \sigma^-  e^{i\delta\omega t} - a \sigma^+  e^{-i\delta\omega t}),
	\label{Hqint}
\end{equation}
где $\hbar g$ --- энергия связи между полем и атомом, $\delta\omega = \omega - \omega_0$ --- отстройка, которую мы считаем достаточно малой, $\sigma^{+} = \ket{e}\bra{g}$ и $\sigma^{-} = \ket{g} \bra{e}$ --- повышающий и понижающий операторы для состояний атома, а $a^\dag$ ($a$) --- операторы рождения и уничтожения фотонных состояний $|n\rangle$  на частоте $\omega = \omega_0 + \delta\omega$. 
Эволюция системы на коротком временном интервале $\Delta t \ll \delta\omega^{-1}$ описывается оператором $U(t',t) = \exp(-\frac{i}{\hbar} H_t \Delta t)$, где $H_t$ --- гамильтониан в момент времени $t$, $\Delta t = t'-t$. Оператор эволюции может быть разложен в бесконечный ряд по формуле:
\begin{equation}
	\begin{split}
		U(t',t) = 1 + 
		\eta (a^\dag s^- - a s^+ ) 
		-\frac{\eta^2}{2!} (a a^{\dag} s_{e} + a^{\dag}a s_{g} ) -
		\frac{\eta^3}{3!} (a^{\dag} a a^\dag s^- - a a^{\dag}a s^+)+... ,
	\end{split}
	\label{ut}
\end{equation}
где $\eta = g\Delta t$ и $s^\pm$ --- зависящие от времени операторы $s^+(t) = \sigma^+ e^{-i\delta\omega t}$ и $s^-(t) = \sigma^- e^{i\delta\omega t}$. 
Данный ряд можно записать в более простом виде: 
\begin{equation}
	\begin{split}
	U(t',t) =  \cos{\big(\eta \sqrt{a^\dag a}\big)} s_g &+ \cos{\big(\eta \sqrt{a a^\dag}\big)} s_e - \\
	&- \frac{a s^+}{\sqrt{a^\dag a}} \sin{\big(\eta \sqrt{a^\dag a}\big)} + \frac{a^\dag s^-}{\sqrt{a a^\dag}} \sin{\big(\eta \sqrt{a a^\dag}\big)},
	\end{split}	
\end{equation}
где $s_e = s^+ s^-$ and $s_g = s^- s^+$. Преобразуя далее, имеем:
\begin{equation}
	\begin{split}
		U(t',t) = \sum_{n=0}^{\infty} \Big[  \cos{\big(\eta \sqrt{n}\big)} &|n\rangle\langle n| s_g + \cos{\big(\eta \sqrt{n+1}\big)} |n\rangle\langle n| s_e - \\ 
		&|n\rangle\langle n+1| s^+ \sin{\big(\eta \sqrt{n}\big)} + |n+1\rangle\langle n| s^- \sin{\big(\eta \sqrt{n+1}\big)} \Big].	
	\end{split}
	\label{ut}
\end{equation}
В частности, для начального состояния $\Psi(t) = \ket{g,n}$, эволюция в результате приводит к состоянию $\Psi(t')=\cos(\eta \sqrt{n}) \ket{g,n}-e^{-i\delta\omega t} \sin(\eta \sqrt{n}) \ket{e, n-1}$. Мы будем рассматривать эволюцию под воздействием сильного когерентного излучения  $\ket{\alpha}$, где действительное $\alpha \gg 1$.

Когда $\Psi(t) = |g,\alpha\rangle$, эволюция упрощается:  
\begin{equation}
	\Psi(t') \approx \cos\frac{\theta}{2} \ket{g, \alpha} - e^{-i\delta\omega t} \sin \frac{\theta}{2} \ket{e, \alpha}
	\label{Rb}
\end{equation}
где $\theta = 2\eta\alpha$, $\alpha = \sqrt{\langle n\rangle}$ и $|\alpha'\rangle = \Big(1-e^{-|\alpha |^2}\Big)^{-1/2}\sum_{n=1}^\infty |n-1\rangle\langle n|\alpha\rangle $. В интересующем нас случае $\alpha \gg 1$ и поэтому поглощение одного фотона атомом эффективно не меняет состояние: $\ket{\alpha'} \approx \ket{\alpha}$. Мы можем переписать состояние в \eqref{Rb} как: 
$\Psi(t') \approx (\cos\frac{\theta}{2} |g\rangle - e^{-i\delta\omega t}\sin \frac{\theta}{2} |e\rangle)\otimes |\alpha\rangle$. После прохождения импульса, фотонное поле полностью излучается ($|\alpha\rangle \rightarrow |0\rangle$) и состояние системы приобретает вид:   
\begin{equation}
	\begin{split}
		\Psi' = \bigg(\cos\frac{\theta}{2} |g\rangle - e^{-i\delta\omega t}\sin \frac{\theta}{2} |e\rangle\bigg)\otimes |0\rangle 	
	\end{split}
	\label{Rb2}
\end{equation}
и для среднего значения оператора понижения с учетом фазы имеем:
\begin{equation}
	\langle s^+ \rangle = -\frac{\sin{\theta}}{2}.
	\label{sp}
\end{equation}

Атом, находящийся в состоянии суперпозиции (при $\theta \neq M \pi$, где $M$ это целое число) приобретает фазу $\delta\omega t$ из начальной когерентной волны \cite{Astafiev2010resonance,abdumalikov2011dynamics} и затем излучает в волновод суперпозицию нуля и одного фотона. 
%This can be exemplified by analysing the evolution $U(t')\Psi' = (\cos\frac{\theta}{2} |g, 0\rangle - ie^{i\phi}(\sin\frac{\theta}{2}\cos\frac{\theta'}{2} |e, 0\rangle - i\sin\frac{\theta}{2}\sin\frac{\theta'}{2} |g, 1\rangle)$, where $\theta' = 4 g t'$. 
Полезно проанализировать эволюцию состояния $\Psi'$ из уравнения \eqref{Rb2} под действием оператора (\ref{ut}). Когда накопленный угол $\eta = \frac{\pi}{2}$, % $U_{ap} = 1-i (s^- a^\dag+s^+ a)$ 
\begin{equation}
	U_{ap} = |0\rangle\langle0| \sigma^-\sigma^+ + e^{i\delta\omega t} |1\rangle\langle 0 | \sigma^-
	\label{Uap}
\end{equation} 
и суперпозиция атомного состояния перетекает в суперпозицию однофотонного поля на частоте $\omega$ следующим образом:    
\begin{equation}
	%\begin{split}
	U_{ap} \bigg[ \bigg(\cos{\frac{\theta}{2}} |g\rangle - e^{-i\delta\omega t} \sin{\frac{\theta}{2}} |e\rangle\bigg) \otimes |0\rangle \bigg] =  
	|g\rangle \otimes \bigg(\cos{\frac{\theta}{2}} |0\rangle - \sin{\frac{\theta}{2}} |1\rangle\bigg).
	\label{sps}
\end{equation} 
Мы введем однофотонный оператор рождения $b^+ = |1\rangle\langle0|$  на частоте $\omega$ и тогда
\begin{equation}
	\langle b^+\rangle = -\frac{1}{2}\sin\theta. 
	\label{bp}
\end{equation}
Уравнения (\ref{sp} - \ref{bp}) теперь могут быть переписаны при помощи $b$-операторов и важным следствием этого является тот факт, что состояние суперпозиции в атоме переходит в когерентное однофотонное поле при замене $s^+ \rightarrow b^+$ и $s^- \rightarrow b^-$, где введен оператор $b^- = |0\rangle\langle 1|$. 

%Importantly, the generated field is coherent, however essentially different from the classical coherent state $|\alpha \rangle=e^{-{|\alpha|^2}/{2}}\sum_{n=0}^{\infty}\alpha^n /\sqrt{n!} |n\rangle$ and $\alpha = \sqrt{\langle n\rangle}$ consisting of an infinite number of photon states. 

Классическое когерентное и однофотонное состояния могут быть представлены в похожей форме: 
\begin{subequations}
	\begin{alignat}{2}
		|&\alpha\rangle = & A \bigg( & |0\rangle + \alpha |1\rangle + \frac{\alpha^2}{\sqrt{2!}} + ...\bigg)\\
		|&\beta\rangle = & B \big( & |0\rangle + \beta |1\rangle\big),
	\end{alignat}
\end{subequations}
где $A=\exp{(-|\alpha|^2/2)}$ и $B=(1+|\beta|^2)^{-1/2}$. В частности, для когерентного состояния из уравнения \eqref{sps}, $\beta = -\tan \theta/2$ и $B=\cos \theta/2$. 

На основе вышеизложенного, гамильтониан~\eqref{Hqint} может быть эквивалентным образом представлен при помощи однофотонных операторов рождения и уничтожения следующим образом: 
\begin{equation}
	H' = i\hbar g (b^+ a - b^- a^\dag),
	\label{Hqintb}
\end{equation}
подразумевая, что $b$-операторы описывают фотоны, излученные атомом с фазой $\delta\omega t$ и потому удовлетворяют соотношениям, типичным для $s$-операторов: $b^+b^- = |1\rangle\langle 1|$, $b^-b^+ = |0\rangle\langle 0|$, $b^+b^+ = 0$, $b^-b^- = 0$. 

В дополнение, можно упростить уравнение~(\ref{ut}) для случая достаточно коротких импульсов достаточно сильного когерентного поля следующим образом: 
%\begin{equation}
%\begin{split}
%	U(t',t) = \sum_{n=0}^{\infty} &\cos{\big(\eta \sqrt{n}\big)} |n\rangle\langle n| b^-b^+ + \cos{\big(\eta \sqrt{n+1}\big)} |n\rangle\langle n| b^+b^- -\\ 
%	&\sin{\big(\eta \sqrt{n}\big)} |n-1\rangle\langle n| b^+ + \sin{\big(\eta \sqrt{n+1}\big)} |n+1\rangle\langle n| b^-.	
%\end{split}
%\label{ut}
%\end{equation}
%
%\begin{equation}
%	U(t',t) = \cos{\big(\eta \sqrt{a^\dag a}\big)} b^-b^+ + \cos{\big(\eta \sqrt{a a^\dag}\big)} b^+b^- -
%	\frac{a b^+}{\sqrt{a^\dag a}} \sin{\big(\eta \sqrt{a^\dag a}\big)} + \frac{a^\dag b^-}{\sqrt{a a^\dag}} \sin{\big(\eta \sqrt{a a^\dag}\big)}.	
%\end{equation}
%
\begin{equation}
	U(t',t) \approx \cos{\big(\eta \sqrt{a^\dag a}\big)} + (a^\dag b^- - a b^+) \frac{\sin{\big(\eta \sqrt{a^\dag a}\big)}}{\sqrt{a^\dag a}}.	
	\label{u1}
\end{equation}

%\begin{equation}
%\langle b^+_{2k+1} \rangle = \frac{(-1)^k}{2}  J_{2k+1}[2\Omega\Delta t]. 
%\end{equation}

Вернемся к случаю импульсной бихроматической накачки. Для этого, мы рассматриваем два непрерывных когерентых поля $|\alpha_{-}\rangle_-$ и $|\alpha_{+}\rangle_+$, где $\alpha_{\pm}$ --- действительные комплексные амплитуды поля. Гамильтониан \eqref{Hqintb} принимает вид:
\begin{equation}
	%H = \hbar g \sum\limits_{k=0}^{\infty} (\sigma^+ a_{\pm(2k+1)} + \sigma^- a^\dag_{\pm(2k+1)}),
	%H = \hbar g (\sigma^+ a_{1} + \sigma^+ a_{-1}+c.c.),
	%H = \hbar g (\sigma^+ a_{1} + \sigma^- a^\dag_{1} + \sigma^+ a_{-1} + \sigma^-a^\dag_{-1}),
	H_{2} = i\hbar g(s_-^-a^\dag_{-} - s_-^+ a_{-}  + s_+^- a^\dag_{+} + s_+^+ a_{+} ) ,
	\label{Hm}
\end{equation}
где $a^\dag_\pm$ ($a_\pm$) --- операторы рождения (уничтожения) фотона на частотах $\omega_\pm$, $s_+^\pm = \sigma e^{\mp i\delta\omega t}$, $s_-^\pm = \sigma e^{\pm i\delta\omega t}$ и $\hbar g$ обозначает константу связи кубита к модам. Оператор эволюции, соответствующий гамильтониану ~(\ref{Hm}) может быть расписан аналогично уравнению~(\ref{ut}), однако, каждое слагаемое содержит последовательные комбинации операторов $s_\pm^- a^\dag_{\pm}$ и $s_\pm^+ a_{\pm}$. Мы можем переписать гамильтониан через $b$-операторы при помощи подстановки $s_\pm^+ \rightarrow b_\pm^+$ and $s_\pm^- \rightarrow b_\pm^-$,  
\begin{equation}
	H = i \hbar g\big(b_-^+ a_{-}  - b_-^-a^\dag_{-} + b_+^+ a_{+} - b^-_+ a^\dag_{+} \big),
	\label{Hmb}
\end{equation}
где $b_\pm^\pm$ описывают возбуждение/релаксацию атома с фазой $\pm\delta\omega t$. Оператор эволюции $U(t',t) = \exp(-\frac{i}{\hbar} H_{t}\Delta t)$ может быть переписан в тензорной форме:  
\begin{equation}
	\begin{split}
		U =1 + \eta (b^-_m a^\dag_m - b^+_m a_m)  &- \frac{\eta^2}{2!}(b^+_m b^-_j a_m a^\dag_j + b^-_m b^+_j a_m^{\dag} a_j) + \\
		&+ \frac{\eta^3}{3!} (b^-_{m-j+p} a^\dag_m a_j a^\dag_p - b^+_{m-j+p} a_m a_j^{\dag} a_p) +...,
	\end{split}
	\label{ut2}
\end{equation}
где индексы принимают значения $\pm 1$. Мы опираемся на соотношения $b^+_m b^-_j b^+_p = b^+_{m-j+p}$ потому что $b$-операторы должны удовлетворять таким же соотношениям, как и $s$-операторы: $s_m^+ s_j^- s_p^+ = e^{-i m\delta\omega t}\sigma^+ e^{i j\delta\omega t}\sigma^- e^{-i p\delta\omega t}\sigma^+ = e^{-i (m-j+p)\delta\omega t}\sigma^+ = s_{m-j+p}^+$. Здесь мы расширили определения  $s$-операторов произвольной $l$-моды: $s_l^\pm = e^{\mp i l\delta\omega t} \sigma^\pm$. К примеру, это означает, что слагаемые третьего порядка $a_+ a^\dag_- a_+ b^+_3$ и $a_- a^\dag_+ a_- b^+_{-3}$ выражаются в создании однофотонных полей на частотах $\omega_{\pm 3} = \omega_0 \pm 3\delta\omega$. Обобщая, излучаемый свет может возникать на частотах $\omega_{\pm l}=\omega_0 \pm l\delta\omega$, где $l=2k+1,k = 0,1,2,..$. Среди всех членов в уравнении~(\ref{ut2}), дающих вклад в создание однофотонного поля на частотах $\omega_{\pm l}$, член низшего порядка состоит из $2k+2$ операторов: $2k+1$ $a$-операторов $a_\pm a^\dag_\mp a_\pm ... = (a_\pm a^\dag_\mp)^k a_\pm$ and один оператор $b_{\pm (2k+1)}^+$. 
%The lowest order term in Supplementary Eq.~(\ref{ut2}) resulting in creation of the single-photon field at frequency $\omega_0 \pm l\delta\omega$, where $l=2k+1$, consists of $2k+2$ operators: $2k+1$ $a$-operators $a_\pm a^\dag_\mp a_\pm ... = (a_\pm a^\dag_\mp)^k a_\pm$ and one $b_{\pm (2k+1)}^+$. %

Как было показано в работе \cite{Astafiev2010resonance}, атом в состоянии суперпозиции геренирует поле $V = \frac{\hbar\Gamma_1}{\mu}\langle s^-\rangle$. Обобщая это утверждение, мы можем записать выражение для однофотонного когеретного поля на частоте $\omega_{\pm (2k+1)}$:
\begin{equation}
	V_{\pm l} = \frac{\hbar\Gamma_1}{\mu}\langle b^-_{\pm (2k+1)}\rangle,
	\label{Vl}
\end{equation}
где $\mu$ --- дипольный момент перехода, в нашем случае зависящий от величины емкостной связи атома с волноводом. Мы делаем замену $s^- \rightarrow b^-_{\pm (2k+1)}$ потому что поле в $b$-моде может быть непосредственно измерено ВАЦ или любым другим АЦП.

%We start from an arbitrary single-photon state with two coherent fields $|\beta\rangle\otimes | \alpha_-\rangle\otimes |\alpha_+\rangle$ and the evolution of the single-photon field averaged out over the driving field degrees of fredom can be described as 
%\begin{equation}
%U_b(t,t') = \langle\alpha_-,\alpha_+| U(t,t') |\alpha_-\alpha_+\rangle. 
%\end{equation}
%We start from a atomic state (mapped on the single-photon field $|\beta\rangle$ as shown above) with two strong coherent driving fields with equal amplitudes $\Psi(t) = |\beta\rangle\otimes | \alpha\rangle_-\otimes |\alpha\rangle_+$, where $|\alpha| \gg 1$. 
Для того, чтобы проанализировать эволюцию, в качестве начального состояния выберем $\Psi(t) = |\beta\rangle\otimes | n\rangle_-\otimes |n\rangle_+$, где атом находится в суперпозиции, описываемой однофотонным когерентным состоянием  $|\beta\rangle$, а поле находится в состояниях $|n\rangle_\pm$ (где $n \gg 1$) с одинаковым числом фотонов в каждой моде.
Введем операторы:  
\begin{subequations}
	\begin{alignat}{2}
		\hat{A}_{2k}^{+-} = &\Bigg\{
		\begin{array}{lr}
			(a^\dag_- a_+)^{-k} & : k < 0\\
			(a^\dag_+ a_-)^k & : k \ge 0
		\end{array}
		&\quad
		\hat{A}_{2k+1}^- = &\Bigg\{
		\begin{array}{lr}
			(a_- a^\dag_+)^{-k} a_- & : k < 0\\
			(a_+ a^\dag_-)^k a_+ & : k \ge 0
		\end{array}
		\\
		\hat{A}_{2k}^{-+} = &\Bigg\{
		\begin{array}{lr}
			(a_- a^\dag_+)^{-k} & : k < 0\\
			(a_+ a^\dag_-)^k & : k \ge 0
		\end{array}
		&\quad
		\hat{A}_{2k+1}^+ = &\Bigg\{
		\begin{array}{lr}
			(a^\dag_- a_+)^{-k} a^\dag_- & : k < 0\\
			(a^\dag_+ a_-)^k a^\dag_+ & : k \ge 0 ,
		\end{array}
	\end{alignat}
\end{subequations}
которые удовлетворяют соотношениям $(\hat{A}^{+})_{2k+1}^\dag = \hat{A}^-_{2k+1}$, $(\hat{A}^{+-}_{2k})^\dag = \hat{A}^{+-}_{-2k}$, $(\hat{A}^{-+}_{2k})^\dag = \hat{A}^{-+}_{-2k}$. 
Оператор
\begin{equation}
	\hat{A}^-_{2k+1} b^+_{2k+1} 
	\label{Ap}
\end{equation}
создает единичный фотон на частоте $\omega_{2k+1}$ с минимально возможным числом фотонов, созданных или уничтоженных на частотах $\omega_\pm$.  
В частности, $A^-_{2k+1} |n_-,n_+\rangle = \big(\frac{(n_- + k)!}{n_-!} \frac{n_+!}{(n_+ - k-1)!} \big)^{\frac{1}{2}} |n_- + k, n_+ - k-1\rangle$, где $k>0$. В случае одинаковых и достаточно больших $n\gg 2k+1$, $A^-_{2k+1}|n,n\rangle \approx n^{k+\frac{1}{2}} |n + k, n - k-1\rangle$. 
Эволюция может быть упрощенно записана в следующей форме: 
\begin{equation}
	%U\Psi = \sum_{k=-\infty}^{\infty} \big[C^{+-}_{2k}\hat{A}_{2k}^{+-} b^-_0 b^+_{2k} + iC^{-}_{2k+1}\hat{A}_{2k+1}^{-} b^+_{2k+1} + C^{-+}_{2k}\hat{A}_{2k}^{-+} b^+_0 b^-_{2k} - iC^{+}_{2k+1}\hat{A}_{2k+1}^{+} b^-_{2k+1}\big] \Psi,
	U(t',t)\Psi(t) \approx \sum_{k=-\infty}^{\infty} \big[\hat{A}_{2k}^{+-} C^{+-}_{2k} b^-_{k} b^+_{-k} + \hat{A}_{2k}^{-+} C^{-+}_{2k} b^+_{k} b^-_{-k} - \hat{A}_{2k+1}^{-} C^{-}_{2k+1} b^+_{2k+1} + \hat{A}_{2k+1}^{+} C^{+}_{2k+1} b^-_{2k+1}\big] \Psi(t),
	%U_b = \sum_{k=-\infty}^{\infty} \big[C_{2k}\langle\hat{A}_{2k}^{+-}\rangle b^-_0 b^+_{2k} + C_{2k+1}\langle\hat{A}_{2k+1}^{-}\rangle b^+_{2k+1} + C_{2k}\langle\hat{A}_{2k}^{-+}\rangle b^+_0 b^-_{2k} + C_{2k+1}\langle\hat{A}_{2k+1}^{+}\rangle b^-_{2k+1}\big]
	\label{uc}
\end{equation}
где коэффициенты $C_l$ зависят от начального состояния и вычисляются как сумма по всем возможным перестановкам комбинаций из операторов рождения и уничтожения ($a^{~}_{-} a^\dag_{-}$, $a^\dag_{-} a^{~}_{-}$, $a^\dag_{+} a^{~}_+$, $a^{~}_{+} a^\dag_{+}$ для двух фотонов, вовлеченных в многофотонный процесс рассеяния, $a^{~}_+ a^{\dag}_- a^{~}_- a^{\dag}_+$,$ a_+^\dag a^{~}_- a_-^\dag a^{~}_+$ и два других члена для 4 фотонов, и т.д.), которые не меняют ни частоту,  ни заселенность фотонных состояний. Предполагая, что $n \gg 2k$ и принимая во внимание соотношения $a\ket{n}=n \ket{n}$, $a^\dag\ket{n}\approx n \ket{n}$, мы приходим к выражению: 
\begin{equation}
	%\frac{\eta^l}{l!}\times \frac{l!}{k! (l-k)!}
	%C_{l} = \frac{1}{\big(a_+^\dag a_+ \big)^{\frac{l}{2}}} \sum_{m=0}^\infty \frac{\bigg(i\eta\sqrt{a_+^\dag a_+}\bigg)^{l+2m}}{(l+2m)!} \frac{(l+2m)!}{m! (l+m)!} = \Bigg(\frac{i}{\sqrt{a^\dag_+ a_+}}\Bigg)^lJ_l \bigg(\eta \sqrt{a^\dag_+ a_+} t \bigg), 
	C_{l} \approx \frac{1}{(\sqrt{n})^l} \sum_{m=0}^\infty \frac{(-1)^{j+m}(\eta  \sqrt{n})^{l+2m}}{(l+2m)!} \frac{(l+2m)!}{m! (l+m)!} = \frac{(-1)^j}{(\sqrt{n})^l} J_l (\eta\sqrt{n}), 
\end{equation}
где $j = \mod(l,2)$, $J_l$ обозначают бесселевские функции первого рода. 

Если начальное состояние системы взять в виде $\Psi = |\beta,\alpha,\alpha\rangle$, где $\alpha$ действительно, то уравнение~(\ref{uc}) упрощается: 
\begin{equation}
	\begin{split}
	\sum_{k=-\infty}^{\infty} \bigg[&\frac{(-1)^k}{\alpha^{2k}}J_{2k}(\theta)\Big(\hat{A}_{2k}^{+-} b^-_{-k} b^+_{k} +\hat{A}_{2k}^{-+} b^+_{-k} b^-_{k} \Big) + \\
	&+ \frac{(-1)^k}{\alpha^{2k+1}}J_{2k+1}(\theta)\Big(\hat{A}_{2k+1}^{+} b^-_{2k+1} - \hat{A}_{2k+1}^{-} b^+_{2k+1}\Big)\bigg] \Psi
	\end{split}
\end{equation}
и для случая $\Psi = |0,\alpha,\alpha\rangle$:
\begin{equation}
	\begin{split}
	U\Psi \approx \sum_{k=-\infty}^{\infty} \bigg[&\frac{(-1)^k}{\alpha^{2k}}J_{2k}(\theta)\hat{A}_{2k}^{+-} |0\rangle_{2k} \otimes|\alpha,\alpha\rangle +\\
	&+ \frac{(-1)^k}{\alpha^{2k+1}}J_{2k+1}(\theta)\hat{A}_{2k+1}^{-} |1\rangle_{2k+1} \otimes|\alpha,\alpha\rangle \bigg], 
	\end{split}
\end{equation}
где $\theta = 2\eta\alpha$. 
%\begin{equation}
%U\Psi = \sum_{k=-\infty}^{\infty} \bigg[\frac{(-1)^k}{\alpha^{2k}}J_{2k}(\Omega t)\hat{A}_{2k}^{+-} e^{i2k\delta\omega t} b^-_{0} b^+_{2k}|0\rangle - \frac{(-1)^k}{\alpha^{2k+1}}J_{2k+1}(\Omega t)\hat{A}_{2k+1}^{-} b^+_{2k+1}|0\rangle \bigg] \Psi,
%\end{equation}
Принимая во внимание, что
$b^+_{2k+1} = |1\rangle_{2(k+p)+1}\langle 0|_{2p}$, мы можем напрямую записать выражение для квантовомеханического среднего оператора рождения фотона на частоте $\omega_{2k+1}$
\begin{equation}
	\langle b^-_{2k+1}\rangle =  \sum_{p=-\infty}^{\infty} \frac{(-1)^{k+p+p}}{\alpha^{2(k+p)+1}}J_{2(k+p)+1}(\theta)J_{2p}(\theta)\langle \hat{A}_{2(k+p)+1}^{-}\hat{A}_{-2p}^{+-}\rangle. 
\end{equation}
Используя стандартные соотношения для функций Бесселя, это выражение упрощается: 
\begin{equation}
	\langle b^-_{2k+1}\rangle = \frac{(-1)^kJ_{2k+1}(2\theta)}{2\alpha^{2k+1}}\langle\hat{A}_{2k+1}^{-}\rangle.
	\label{b22}
\end{equation}
Здесь было использовано следующее свойство: $\alpha^{-(2(k+p)+1)} \langle \hat{A}_{2(k+p)+1}^{-}\hat{A}_{-2p}^{+-} \rangle \approx\alpha^{-(2k+1)}\langle \hat{A}_{2k+1}^{-} \rangle$ for $\alpha \gg 1$.
Учитывая, что $\langle \hat{A}^-_{2k+1} \rangle \approx \alpha^{2k+1}$, мы упрощаем выражение:
\begin{equation}
	\langle b^+_{2k+1}\rangle = \frac{(-1)^k}{2} J_{2k+1}(2\Omega \Delta t),  
	\label{b2k1}
\end{equation}
где $\Omega \Delta t = \theta$. Окончательный ответ совпадает с  (\ref{sclass_spectr}).
Амлитуда когерентно рассеиваемого в каждую моду напряжения может быть записана как:  
\begin{equation}
	V_{2k+1} = \frac{\hbar \Gamma_1}{\mu}\langle b_{2k+1} \rangle, 
\end{equation}
что соотвествует мощности 
\begin{equation}
	W_{2k+1} = \frac{V^2_{2k+1}}{2Z_0} , 
\end{equation}
где $Z_0$ --- волновой импеданс. Теперь мы рассчитаем энергию когерентной волны за каждый цикл, подставляя уравнение~(\ref{b2k1}) в~(\ref{Vl}) и производя интегрирование по времени $t$. Учитывая, что $\Gamma_1 = \frac{\hbar\omega \mu^2 Z_0}{\hbar^2}$ и $\int_0^\infty \langle b_{2k+1} \rangle^2 dt = \int_0^\infty e^{-\Gamma_1t} = \Gamma_1^{-1}$, мы находим числа фотонов, излучаемых суммарно в две половины волновода:
\begin{equation}
	%V_{\pm(2k+1)} = \frac{\hbar\Gamma_1}{\mu}\frac{(-1)^{k+1}}{2}\times J_{\pm(2k+1)}(2\Omega \Delta t).
	\frac{E_{\pm(2k+1)}}{\hbar\omega} = \frac{J^2_{\pm(2k+1)}(2\Omega \Delta t)}{4}.
\end{equation} 

Несмотря на то, что аналитическое решение получено в приближении сильного драйва, его можно обобщить и на случай произвольных начальных состояний поля:
\begin{equation}
	\langle b^+_{2k+1}\rangle = D^-_{2k+1}\langle\hat{A}_{2k+1}^{-}\rangle,
	%\langle b^-_{2k+1}\rangle = B^-_{2k+1}\langle\hat{A}_{2k+1}^{+}\rangle. 
	\label{bg}
\end{equation}
где коэффициенты $D^-_{2k+1}$ зависят от амплитуд состояний поля. Также мы видим, что фотоны испускаются только на частотах $\omega_0 \pm (2k+1)\delta\omega$ с нечетными индексами $2k+1 > 0$. Создание такого однофотонного состояния требует уничтожения $k+1$ фотонов на частоте $\omega_+$ и создания $k$ фотонов на частоте $\omega_-$. 

В данном разделе мы теоретически и экспериментально рассмотрели бесселевский аналог осцилляций Раби, возникающий в рассеянном поле на боковых частотах при волновом смешении двух синхронных последовательностей импульсов на одиночном искусственном атоме. Теперь мы обратимся к эффекту, называемому квантовым волновым смешением, который проявляется в несимметрии эластичного спектра при рассеянии последовательности бихроматических импульсов с задержкой. Его необычность в том, что он может наблюдаться только при рассеянии на одиночном атоме, поэтому ранее он не был обнаружен в рамках традиционной квантовой оптики на ансамблях двухуровневых систем. 
\section{Введение задержки. Квантовое смешение волн}
\section{Квантовое смешение волн на 3-уровневой системе}
\section{Численный расчет импульсной динамики}
\section{Аналитический расчет в представлении вторичного квантования}

