\chapter*{Список сокращений и условных обозначений}             % Заголовок
\addcontentsline{toc}{chapter}{Список сокращений и условных обозначений}  % Добавляем его в оглавление
\noindent
\begin{longtabu} to \textwidth{X[r] X[1] }
%\begin{longtabu} to\dimexpr \textwidth-5\tabcolsep {r X}
% Жирное начертание для математических символов может иметь
% дополнительный смысл, поэтому они приводятся как в тексте
% диссертации
T & температура \\
T_c & критическая темература сверхпроводящего перехода \\
\Gamma^{(c,e)}_1 & время излучательной релаксации (верхний индекс указывает линию, в которую излучает кубит) \\
\Gamma_1^{nr} & время безызлучательной релаксации \\

% $\begin{rcases}
%a_n\\
%b_n
%\end{rcases}$  & 
%\begin{minipage}{\linewidth}
%коэффициенты разложения Ми в дальнем поле соответствующие
%электрическим и магнитным мультиполям
%\end{minipage}
%\\
% ${\boldsymbol{\hat{\mathrm e}}}$ & единичный вектор \\
% $E_0$ & амплитуда падающего поля\\
% $\begin{rcases}
%a_n\\
%b_n
%\end{rcases}$  & 
%коэффициенты разложения Ми в дальнем поле соответствующие
%электрическим и магнитным мультиполям ещё раз, но~без окружения
%minipage нет вертикального выравнивания по~центру.
%\\
% $j$ & тип функции Бесселя\\
% $k$ & волновой вектор падающей волны\\
%
% $\begin{rcases}
%a_n\\
%b_n
%\end{rcases}$  & 
%\begin{minipage}{\linewidth}
%\vspace{0.7em}
%и снова коэффициенты разложения Ми в дальнем поле соответствующие
%электрическим и магнитным мультиполям, теперь окружение minipage есть
%и добавлено много текста, так что описание группы условных
%обозначений значительно превысило высоту этой группы... Для отбивки
%пришлось добавить дополнительные отступы.
%\vspace{0.5em}
%\end{minipage}
%\\
% $L$ & общее число слоёв\\
% $l$ & номер слоя внутри стратифицированной сферы\\
% $\lambda$ & длина волны электромагнитного излучения
%в вакууме\\
% $n$ & порядок мультиполя\\
% $\begin{rcases}
%{\mathbf{N}}_{e1n}^{(j)}&{\mathbf{N}}_{o1n}^{(j)}\\
%{\mathbf{M}_{o1n}^{(j)}}&{\mathbf{M}_{e1n}^{(j)}}
%\end{rcases}$  & сферические векторные гармоники\\
% $\mu$  & магнитная проницаемость в вакууме\\
% $r,\theta,\phi$ & полярные координаты\\
% $\omega$ & частота падающей волны\\

\textbf{SIS} & Superconductor-Insulator-Superconductor, сверхпроводник-изолятор-сверхпроводник \\
\textbf{JJ} & Josephson Junction, переход Джозефсона \\
\textbf{сQED} & Квантовая электродинамика на основе электрических цепей \\
\textbf{СКЦ} & Сверхпроводящие квантовые цепи \\
\textbf{СВЧ} & Сверх-высокочастотный \\
\textbf{(вч-)СКВИД} & (высокочастотный) сверхпроводящий квантовый интерферометр \\
\textbf{RCSJ} & Модель резистивно и емкостно шунтированного перехода \\
\textbf{ВАХ} & Вольт-амперная характеристика \\
\textbf{IPA} & Изопропанол \\
\textbf{NMP} & Несимметричный метил-пиролидон \\
\textbf{RCA-1} & Стандартный раствор NH$_3$, Н$_2$О$_2$ в Н$_2$O, используемый для первичной очистки кремниевых пластин \\
\textbf{СЭМ} & Сканирующий туннельный микроскоп \\
\textbf{SMP} & Тип высокочастотных коаксиальных разъемов \\ 
\textbf{Still} & Фланец криостата с температурой порядка 700 мК \\
\textbf{ОСШ} & Отношение <<сигнал-шум>> \\
\textbf{ПЧ} & Промежуточная частота \\
\textbf{LO} & Несущая частота \\
\textbf{АЦП} & Аналогово-цифровой преобразователь \\
\textbf{СА} & Спектральный анализатор \\
\textbf{ВАЦ} & Векторный анализатор цепей \\
\textbf{RBW} & Полоса разрешения спектрального анализатора \\
\textbf{I} & компонента поля, находящаяся в фазе с некоторым опорным сигналом \\
\textbf{Q} & компонента поля, перпендикулярная некоторому опорному сигналу \\
\textbf{ГИПФ} & Генератор импульсов произвольной формы \\
\textbf{QuTiP} & Quantum Toolbox in Python --- библиотека для проведения квантовомеханических расчетов на языке программирования \textit{Python} \\
\end{longtabu}
\addtocounter{table}{-1}% Нужно откатить на единицу счетчик номеров таблиц, так как предыдующая таблица сделана для удобства представления информации по ГОСТ
