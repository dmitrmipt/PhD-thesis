\chapter{Двухчастотное волновое смешение на кубите: случай непрерывных волн}
Первый оригинальный эксперимент заключается в изучении волнового смешения на двухуровневой системе, играющей роль искусственного атома в открытом пространстве. Прежде чем перейти конкретно к выполненному в рамках работы эксперименту --- рассеянию двух почти резонансных микроволн на кубите --- рассмотрим основные принципы волнового смешения в том классическом виде, в котором оно описывается в нелинейной оптике сплошных сред для случая смешивания на ансамблях двухуровневых систем.
\section{Оптическое волновое смешение на двухуровневых системах}
Рассматриваемый в классических учебниках по нелинейной оптике случай волнового смешения сводится к изучению поведения двухуровневой системы под действием двух взаимодействующих с ней волн:
\begin{equation}
\tilde{E}(t) = Ee^{-i\omega t} = E_0e^{i\omega t} + E_1e^{i(\omega+\delta) t},
\label{eq: field}
\end{equation}
то есть, под действием волны накачки c произвольной (и возможно, достаточно большой) амплитудой $E_0$ ,и отстроенной от накачки пробной волны малой амплитуды $E_1 \ll E_0$. Уравнения Блоха для двухуровневой системы могут быть записаны для заселенности кубита $z = \rho_{11}-\rho_{00}$ и для поляризации кубита $p=\mu_{01}\rho_{10}$ в следующем виде:
\begin{equation}
\systeme{\frac{dp}{dt} = \left(\delta-i\Gamma_2\right)p -\frac{i}{\hbar}|\mu_{01}|^2Ez,
	 \frac{dz}{dt} = -(z-z^{(eq)})\Gamma_1 + \frac{4}{\hbar}\text{Im}(pE^*)}
\end{equation}
Внешнее поле \eqref{eq: field} таково, что решение уравнений необходимо искать в виде:
\begin{align}
	p =& p_0 + p_1 e^{-i\delta t} + p_{-1}e^{i\delta t}, \\
	z =& z_0 + z_1 e^{-i\delta t} + z_{-1}e^{i\delta t},
\label{eq: sol_bloch_WM}
\end{align}
предполагая при этом, что $|p_0| \gg |p_1|, |p_{-1}|$~и~$z_0 \gg z_1, z_{-1}$.  При сделанном предположении о малости дополнительных компонент, решение этих уравнений прямолинейно, но достаточно громоздко. В результате получаются выражения для каждой из компонент, входящих в \eqref{eq: sol_bloch_WM}, в первом порядке малости по $p_{\pm 1}, z_{\pm 1}$. Для дальнейшего анализа можно использовать нелинейный аналог уравнений Гольмгельца (см. \cite{boyd2003nonlinear}, Глава 2) на частотные компоненты комплексных амплитуд поля:
\begin{equation}
\left( \frac{\partial^2}{\partial x^2} + k_n^2\right) \mathbf{E}_n(\mathbf{r}) = -\frac{k_n^2}{\varepsilon\varepsilon_0}\mathbf{P}^{NL}_n(\mathbf{r})
\end{equation} 
\section{Боковые гармоники в спектре эластично рассеянного сигнала}
кратко поясняем почему они есть, мб ссылки на литературу
\section{Приближение малой отстройки}
прелюдия к расчету: пики от прецессии аналитического решения
\section{Аналитическое выражение для амплитуд боковых гармоник}
расчет и сопоставление
\section{Численное решение уравнений Максвелла-Блоха}
красивая большая картинка например про режим $\Omega \approx \delta \omega$
\section{Случай несбалансированных амплитуд}
 ну случай и случай