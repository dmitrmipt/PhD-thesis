\chapter{Двухчастотное волновое смешение на кубите: случай непрерывных волн}
Первый оригинальный эксперимент заключается в изучении волнового смешения на двухуровневой системе, играющей роль искусственного атома в открытом пространстве. Прежде чем перейти конкретно к выполненному в рамках работы эксперименту --- рассеянию двух почти резонансных микроволн на кубите --- рассмотрим основные принципы волнового смешения в том классическом виде, в котором оно описывается в нелинейной оптике сплошных сред для случая смешивания на ансамблях двухуровневых систем.
\section{Оптическое волновое смешение на двухуровневых системах}
Рассматриваемый в классических учебниках по нелинейной оптике случай волнового смешения сводится к изучению поведения двухуровневой системы под действием двух взаимодействующих с ней волн: волны накачки $E_0e^{i\omega t}$ c произвольной (и возможно, достаточно большой) амплитудой $E_0$ ,и отстроенной от накачки пробной волны малой амплитуды $E_1e^{i(\omega+\delta) t}$. Уравнения Блоха для двухуровневой системы могут быть записаны для заселенности кубита $z = \rho_{11}-\rho_{00}$ и для поляризации кубита $p=\mu_{01}\rho_{10}$ в следующем виде:
\begin{equation}
1	
\end{equation}
\section{Боковые гармоники в спектре эластично рассеянного сигнала}
кратко поясняем почему они есть, мб ссылки на литературу
\section{Приближение малой отстройки}
прелюдия к расчету: пики от прецессии аналитического решения
\section{Аналитическое выражение для амплитуд боковых гармоник}
расчет и сопоставление
\section{Численное решение уравнений Максвелла-Блоха}
красивая большая картинка например про режим $\Omega \approx \delta \omega$
\section{Случай несбалансированных амплитуд}
 ну случай и случай