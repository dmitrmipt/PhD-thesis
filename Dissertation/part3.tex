\chapter{Двухчастотное волновое смешение на кубите: случай непрерывных волн}
Первый оригинальный эксперимент заключается в изучении волнового смешения на кубите. 
краткое введение и описание результатов
\section{Боковые гармоники в спектре эластично рассеянного сигнала}
кратко поясняем почему они есть, мб ссылки на литературу
\section{Приближение малой отстройки}
прелюдия к расчету: пики от  прецессии аналитического решения
\section{Аналитическое выражение для амплитуд боковых гармоник}
расчет и сопоставление
\section{Численное решение уравнений Максвелла-Блоха}
красивая большая картинка например про режим $\Omega \approx \delta \omega$
\section{Случай несбалансированных амплитуд}
 ну случай и случай