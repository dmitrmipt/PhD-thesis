\chapter{Двухчастотное волновое смешение на кубите: случай непрерывных волн}
Основные экспериментальные результаты работы заключается в изучении волнового смешения на двухуровневой системе, играющей роль искусственного атома в открытом пространстве. Прежде чем перейти конкретно к выполненному в рамках работы эксперименту --- рассеянию двух почти резонансных микроволн на кубите --- рассмотрим основные принципы волнового смешения в том классическом виде, в котором оно описывается в нелинейной оптике сплошных сред для случая смешивания на ансамблях двухуровневых систем.
\section{Оптическое волновое смешение на двухуровневых системах}
Рассматриваемый в имеющейся литературе по нелинейной оптике случай волнового смешения является одним из примеров большого класса нелинейных параметрических процессов --- процессов генерации новых частотных компонент или изменения исходных компонент при распространении света в нелинейной среде. Такие процессы удобно описывать при помощи волнового уравнения, учитывающего нелинейную поляризацию среды $\mathbf{P}^{N\!L}$. В классическом учебнике \cite{boyd2003nonlinear} показано, что данное уравнение можно записать для каждой частотной компоненты поля в виде:
\begin{equation}
\left( \frac{\partial^2}{\partial z^2} + k_n^2\right) \mathbf{E}_n(\mathbf{r}) = -\frac{k_n^2}{\varepsilon\varepsilon_0}\mathbf{P}^{N\!L}_n(\mathbf{r}),
\label{eq: P_NL}
\end{equation}
где роль <<внешнего силы>> играет нелинейная часть электрической поляризации $\mathbf{P}^{N\!L}_n(\mathbf{r}) = \mathbf{P}_n(\mathbf{r}) - \mathbf{P}^{1}_n(\mathbf{r})$. Для ансамбля двухуровневых систем поляризацию можно рассчитать, используя решения динамических уравнений Блоха, рассмотренных в разделе \ref{sec: scatt}. Уравнения Блоха для двухуровневой системы могут быть записаны для заселенности кубита $z = \rho_{11}-\rho_{00}$ и для поляризации кубита $p=\mu_{01}\rho_{10}$ в следующем виде:
\begin{equation}
\systeme{\frac{dp}{dt} = \left(\Delta-i\Gamma_2\right)p -\frac{i}{\hbar}|\mu_{01}|^2Ez,
	\frac{dz}{dt} = -(z-z^{(eq)})\Gamma_1 + \frac{4}{\hbar}\text{Im}(pE^*)},
\label{eq: bloch_p_z}
\end{equation}
где $\Delta = \omega - \omega_{01}$.
В рассматриваемом случае нас интересует поведение двухуровневой системы под действием двух взаимодействующих с ней волн:
\begin{equation}
\tilde{E}(t) = Ee^{-i\omega t} = E_0e^{i\omega t} + E_1e^{i(\omega+\delta) t},
\label{eq: field}
\end{equation}
то есть, под действием волны накачки c произвольной (и возможно, достаточно большой) амплитудой $E_0$, и отстроенной от накачки на величину $\delta$ пробной волны малой амплитуды $E_1 \ll E_0$. Внешнее поле \eqref{eq: field} таково, что решение уравнений \eqref{eq: bloch_p_z} необходимо искать в виде:
\begin{align}
p =& p_0 + p_1 e^{-i\delta t} + p_{-1}e^{i\delta t}, \\
z =& z_0 + z_1 e^{-i\delta t} + z_{-1}e^{i\delta t},
\label{eq: sol_bloch_WM}
\end{align}
предполагая при этом, что $|p_0| \gg |p_1|, |p_{-1}|$~и~$z_0 \gg z_1, z_{-1}$.  При учете только первого порядка малости дополнительных компонент, решение этих уравнений прямолинейно, но достаточно громоздко. Опуская подробности, приведем результат решения \eqref{eq: bloch_p_z} для заселенности на частоте драйва $z_0$, заселенности на пробной частоте $z_1$ и поляризации на пробной частоте $p_{\m}$, и наконец, поляризации $p_{-1}$ на  комбинационной частоте $\omega-\delta = 2\omega - (\omega+\delta)$:
\begin{align}
z_0 = & - {{1+\Delta^2/\Gamma_2^2} \over {1+ \Delta^2/\Gamma^2_2 + \Omega^2/\Gamma_1\Gamma_2}}, \\
z_1 = &  - z_0\Omega^2{{E_1} \over {2E_0}} \frac{(\delta-\Delta + i\Gamma_2)(\delta+2i\Gamma_2)}{(\Delta-i\Gamma_2)D(\delta)}, \\
p_1 = & \frac{|\mu_{01}|^2z_0E_1}{\hbar D(\delta)}\left[\left(\delta + i\Gamma_1\right)\left(\delta-\Delta+i\Gamma_2\right)-\frac{\delta\Omega^2}{2(\Delta-i\Gamma_2)}\right] \\
p_{\m} = & 2 z_0 \frac{|\mu_{01}|^4E_0^2E_1^*}{\hbar^3 D^*(\delta)}\frac{(\delta-\Delta-i\Gamma_2)(-\delta+2i\Gamma_2)}
{(\Delta + i\Gamma_2)(\Delta-\delta+i\Gamma_2)}.
\label{eq: sol_bloch_mixing}
\end{align}

Каждую частотную компоненту поля $E_n$ можно записать в виде:
\begin{equation}
E_n = A_n e^{-i(\omega_nt - k_nz)} + \text{c.c.},
\label{eq: E_complex}
\end{equation}
где мы ввели комплексную амплитуду $A_n$, не меняющуюся во времени.  Для простоты будем считать, что среда обладает только нелинейностью третьего порядка, и среди всех параметрических процессов этого порядка нас интересует компоненты, возникающие за счет вклада в поляризацию на частоте $\omega_{-1} = 2\omega_0-\omega_1$ следующего вида:
\begin{equation}
\tilde{P}^{N\!L} = {P}_{-1} e^{-i(2\omega_1-\omega_0)t} + \text{c.c.}= 3\varepsilon_0\chi^{(3)}E_0^2E_1^*e^{-i(2\omega_1-\omega_0)t} + \text{c.c.}.
\end{equation} 
После подстановки в это уравнение выражений \eqref{eq: E_complex} для компоненты на боковой частоте имеем:
\begin{equation}
P_{-1} = 3\varepsilon_0\chi^{(3)}A_0^2A_1^*e^{i(2k_0-k_1)z}.
\end{equation}
Имея выражение для поляризации на нужной частоте, можем записать \eqref{eq: P_NL} для комплексной амплитуды поля, возникшего за счет смешения:
\begin{equation}
ку \left[ \frac{\partial^2}{\partial z^2} A_{\m} - 2ik_{\m}\frac{\partial}{\partial z}A_{\m}\right] + \text{ c.c.} = 3\frac{\chi^{(3)}k^2_{\m}}{\varepsilon} A_0^2A_1^* e^{i(2k_0-k_1-k_{\m})z} + \text{ c.c.}
\end{equation}
Считая, что амплитуда медленно меняется во времени, можно пренебречь слагаемым $\partial^2A_{\m}/\partial z^2$. Далее, поскольку эффект связывает слабые поля на частотах $\omega_1$ и $\omega_{\m}$, то необходимо проделать всю процедуру для поля с амплитудой $A_1$. В итоге получается система связанных уравнений для комплексных амплитуд:
\begin{equation}
\systeme{
\frac{\partial A_1}{\partial z} = -\alpha_1 A_1 + \kappa_1A^*_{\m} e^{i\Delta k z},
\frac{\partial A_{\m}}{\partial z} = -\alpha_{\m} A_{\m} + \kappa_{\m}A^*_1 e^{i\Delta k z}
}
\end{equation}
\section{Боковые гармоники в спектре эластично рассеянного сигнала}
кратко поясняем почему они есть, мб ссылки на литературу
\section{Приближение малой отстройки}
прелюдия к расчету: пики от прецессии аналитического решения
\section{Аналитическое выражение для амплитуд боковых гармоник}
расчет и сопоставление
\section{Численное решение уравнений Максвелла-Блоха}
красивая большая картинка например про режим $\Omega \approx \delta \omega$
\section{Случай несбалансированных амплитуд}
 ну случай и случай