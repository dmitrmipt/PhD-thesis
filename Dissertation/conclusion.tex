\chapter*{Заключение}						% Заголовок
\addcontentsline{toc}{chapter}{Заключение}	% Добавляем его в оглавление

%% Согласно ГОСТ Р 7.0.11-2011:
%% 5.3.3 В заключении диссертации излагают итоги выполненного исследования, рекомендации, перспективы дальнейшей разработки темы.
%% 9.2.3 В заключении автореферата диссертации излагают итоги данного исследования, рекомендации и перспективы дальнейшей разработки темы.
%% Поэтому имеет смысл сделать эту часть общей и загрузить из одного файла в автореферат и в диссертацию:

Основные результаты работы заключаются в следующем:
%% Согласно ГОСТ Р 7.0.11-2011:
%% 5.3.3 В заключении диссертации излагают итоги выполненного исследования, рекомендации, перспективы дальнейшей разработки темы.
%% 9.2.3 В заключении автореферата диссертации излагают итоги данного исследования, рекомендации и перспективы дальнейшей разработки темы.
\begin{enumerate}
  \item Получены и исследованы образцы потоковых кубитов, сильно связанных с континумом электромагнитных мод;
  \item Изучены процессы смешивания классических волн на двухуровневом искусственном атоме;
  \item Продемонстрирован и описан режим квантового смешивания волн; \ldots
  \item Продемонстрировано и описано трехволновое смешивание на трехуровневой системе с $\Delta$-конфигурацией переходов. 
\end{enumerate}

Таким образом, в диссертационной работе эксперименально и теоретически проанализирован ряд квантовооптических эффектов, проявляющихся при взаимодействии когерентного света с одиночной сверхпроводниковой квантовой схемой --- искусственным атомом. Автор надеется, что результаты работы будут полезны в дальнейшем развитии квантовой микроволновой фотоники и оптики на основе искусственных атомов.

В первую очередь, автор выражает глубочайшую благодарность профессору Астафьеву Олегу Владимировичу за исключительно профессиональное, ответственное, доброжелательное и чуткое руководство научной работой, за готовность обсуждать самые нелепые и неоднозначные вопросы автора в любое время суток, за помощь в освоении непростой квантовой физики, за доверие и помощь. Автор также благодарит Логинову Елену Николаевну за огромный вклад в организацию работы лаборатории Искусственных Квантовых Систем МФТИ, за большой жизненный и научный опыт и готовность бесконечно делиться им с молодыми сотрудниками. Автор благодарит Рязанова Валерия Владимировича за поддержку в различных вопросах, за организационную заботу, за внимательное отношение к молодым сотрудникам и за исключительную доброту и широту души. Автор благодарит Федорова Глеба Петровича за готовность обсуждать самые разные вопросы науки и философии, а также за хорошее чувство юмора и за дружеское отношение. Автор благодарит Шайхайдарова Раиса, Антонова Владимира и Терезу Хонигль-Декринис за гостеприимство и  заботу, а также за помощь в освоении азов нанофабрикации. Автор благодарит Коростылева Евгения, Киртаева Романа, Егорова Сергея и Негрова Дмитрия за огромную работу по поддержанию ЦКП МФТИ в рабочем состоянии, без чего невозможно было бы достичь результатов, представленных в диссертации. Автор благодарит аспиранта Кадырметова Шамиля, студентов Васенина Андрея, Гунина Сергея. за возможность передать им свой скромный опыт и за снисходительность к недостаткам автора как научного руководителя.  Автор благодарит весь коллектив лаборатории ИКС МФТИ, в частности Храпача И.Н., Болгара А.Н., Калачеву Д.С., Гунина С.А., Стрельникова А.С., Воробьеву С.Н, Кулакову А.И., Зотову Ю.В., Юрса В.Б., Сандуляну Ш.В., и крайне ценит возможность общения и совместной работы с каждым из своих коллег.  Автор благодарит сотрудников лаборатории сверхпроводящих метаматериалов МИСиС Беседина И.С., Абрамова Н.В., Чичкова В.В. и руководителя лаборатории проф. Устинова А.В. за гостеприимство, за готовность к коллаборации, за возможность одолжить необходимые микроволновые компоненты и всяческую поддержку любой полезно деятельности. Наконец, автор бесконечно благодарен своим родителям Юрию Викторовичу и Светлане Васильевне, любимой супруге Анне и своим детям Елизавете, Василисе, Федору и Вере за любовь, терпение, поддержку и понимание, без которых диссертация не была бы завершена. 

