\chapter*{Заключение}						% Заголовок
\addcontentsline{toc}{chapter}{Заключение}	% Добавляем его в оглавление

%% Согласно ГОСТ Р 7.0.11-2011:
%% 5.3.3 В заключении диссертации излагают итоги выполненного исследования, рекомендации, перспективы дальнейшей разработки темы.
%% 9.2.3 В заключении автореферата диссертации излагают итоги данного исследования, рекомендации и перспективы дальнейшей разработки темы.
%% Поэтому имеет смысл сделать эту часть общей и загрузить из одного файла в автореферат и в диссертацию:

Основные результаты работы заключаются в следующем:
%% Согласно ГОСТ Р 7.0.11-2011:
%% 5.3.3 В заключении диссертации излагают итоги выполненного исследования, рекомендации, перспективы дальнейшей разработки темы.
%% 9.2.3 В заключении автореферата диссертации излагают итоги данного исследования, рекомендации и перспективы дальнейшей разработки темы.
\begin{enumerate}
  \item Спроектированы и изготовлены и исследованы образцы потоковых кубитов, сильно связанных с континумом электромагнитных мод как в геометрии боковой связи (кубит в линии), так и в геометрии прямой связи (кубит, асимметрично связанный с двумя полупространствами);
  \item Воспроизведены базовые квантовооптические эксперименты с одиночными кубитами --- в частности, измерена однотоновая спектроскопия в зависимости от внешнего магнитного поля, зависимость формы линии кубита от амплитуды накачки, измерен триплет Моллоу, приведены результаты различных импульсных измерений, в частности осцилляции Раби и свободное затухание Рамзи;
  \item Получен \textit{эффект непрерывного волнового смешения} на кубите в линии. Показано, что эластичная часть спектра излучения, образовывающегося при рассеянии на кубите двух непрерывных монохроматических волн, несущие частоты которых $\omega_+, \omega_-$ отличается от частоты кубита на $\delta\omega \ll \Gamma_1$, состоит из большого количества пиков аппаратной ширины на частотах $\omega_{\pm(2p+1)}= (p+1)\omega_{\pm}-p \omega_{\mp}$.  При этом возможность наблюдения пиков порядка $p$ ограничено только наличием шума усилителя, используемого в измерительном тракте;
  \item Получена формула, которая описывает амплитуду каждого из вышеупомянутых когерентных компонент (пиков) $\Omega^{sc}_{\pm(2p+1)}$ в зависимости от амплитуды волн накачки $\Omega_+, \Omega_-$ и остальных параметров кубита. Проверено, что экспериментально измеренные амплитуды соответствуют полученной формуле как в случае $\Omega_- = \Omega _+ = \Omega$, так и в случае $\Omega_+ \ne \Omega_-$, причем амплитуды боковых пиков исключительно чувствительны к неравентству амплитуд волн накачки;
  \item Показано, что отношение амплитуд двух соседних пиков, отвечающих отличающимся на единицу значениям $p$, не зависит от $p$, а определяется только параметрами системы. 
  \item Показан эффект, аналогичный расщеплению Аутлера-Таунса и наблюдающийся для боковых компонент при синхронном увеличении эффективной частоты Раби каждой из волн накачки. Показано, что величина расщепления зависит от порядка смешения и выражается соотношением $2\Delta\omega = 8\Omega/(2p+1)$.
  \item Приведены аргументы в пользу того, что эффект непрерывного волнового смешения может использоваться для определения фотонной статистики стационарного или квазистационарного поля в волноводе;
  \item Исследован эффект \textit{импульсного волнового смешения} на кубите в линии. Показано, что при рассеивании на кубите последовательности коротких импульсов длительностью $\Delta t \ll 1/\Gamma_1$ с несущими частотами $\omega_+, \omega_-$, определенными выше, попадающими на кубит одновременно (без задержки) с периодом $T_r \gg 1/\Gamma_1$ наблюдается бесселевская динамика в зависимостях $\Omega^{sc}_{\pm(2p+1)}(\Omega \Delta t)$, где $\Omega=\Omega_+=\Omega_-$ --- амплитуда волн накачки. Пренебрегая затуханием, получена точная формула \eqref{Bessel_power} для энергии в числе фотонов на время жизни кубита, излучаемой в каждой из боковых компонент, которая описывает экспериментальные результаты без подгоночных параметров. 
  \item Исследован эффект \textit{квантового волнового смешения} на кубите в линии. Показано, что если ввести достаточно большую задержку между импульсами с различными частотами (настолько большую, чтобы импульсы не перекрывались во времени), то спектр эластичного рассеяния модифицируется особенным образом: остается единственный боковой пик на частоте $\omega_{-3}$, если импульс на частоте $\omega_-$ следует за импульсом на частоте $\omega_+$, либо же пик на частоте $\omega_+$, если импульс $\omega_+$ следует за импульсом на частоте $\omega_-$. Предложено качественное объяснение наблюдаемому эффекту: кубит может <<запомнить>> только единственный квант возбуждения, другими словами --- поглотить единственный фотон из первого импульса, что запрещает все процессы многофотонного рассеяния, кроме единственного процесса на частоте $2\omega_--\omega_+$
  \item Получены аналитические выражения для зависимости амплитуды пиков на частотах $\omega_+, \omega_- \text{и} \omega_{+3}$ от эффективного угла поворота $\Omega\Delta t$.
  \item Исследован эффект квантового волнового смешения на \textit{трехуровневой эквидистантной квантовой системе}, которой является потоковый кубит при определенном значении внешнего магнитного потока. Показано, что при облучении неперекрывающимися импульсами возникают боковые пики на частотах $\omega_{+5}, \omega_{+3} \text{ и } \omega_{-3}$. Качественная интерпретация эффекта состоит в том, что трехуровневая система может находится в состоянии, где число возбуждений равно 2, и таким образом становятся разрешенными те процессы, где число фотонов из импульса на частоте $\omega_-$ не превышает 2.
  \item Построена модель трехуровневой эквидистантной системы, возбуждаемой классическим полем, в рамках этой модели получены аналитические зависимости амплитуд боковых компонент от эффективного угла поворота $R\Omega \Delta t$, где $R$ зависит от дипольного момента верхнего перехода системы. 
  \item Впервые экспериментально получено и исследовано трехволновое смешение на $\Delta-$системе в трех возможных режимах, когда осуществляется резонансная накачка двух переходов и изучается когерентно рассеянный сигнал на частоте третьего перехода. Показано, что экспериментальные результаты во всех режимах хорошо согласуются как с аналитическим решением основного квантового уравнения, описывающего динамику системы, так и с численным решением.  
  \item Проведено экспериментальное исследование потокового кубита, асимметрично связанного с двумя полупространствами. Показано, что двухтоновая спектроскопия позволяет увидеть рассеянное поле и восстановить спектр кубита до третьего возбужденного уровня включительно. Для конкретного образца проведена оценка эффективности генерации одиночных фотонов, которая составила 70\%. Также изучено расщепление Аутлера-Таунса на трехуровневой системе и показано, что максимум когерентного излучения наблюдается в случае, когда Раби-частота накачки совпадает с константой релаксации накачиваемого перехода.
\end{enumerate}

Таким образом, в диссертационной работе эксперименально и теоретически проанализирован ряд квантовооптических эффектов, проявляющихся при взаимодействии когерентного света с одиночной сверхпроводниковой квантовой схемой --- искусственным атомом. Автор надеется, что результаты работы будут полезны в дальнейшем развитии квантовой микроволновой фотоники и оптики на основе искусственных атомов.

В первую очередь, автор выражает глубочайшую благодарность профессору Астафьеву Олегу Владимировичу за исключительно профессиональное, ответственное, доброжелательное и чуткое руководство научной работой, за готовность обсуждать самые нелепые и неоднозначные вопросы автора в любое время суток, за помощь в освоении непростой квантовой физики, за доверие и поддержку. Автор также благодарит Логинову Елену Николаевну за огромный вклад в организацию работы лаборатории Искусственных Квантовых Систем МФТИ, за большой жизненный и научный опыт и готовность бесконечно делиться им с молодыми сотрудниками. Автор благодарит Рязанова Валерия Владимировича за поддержку в различных вопросах, за организационную заботу, за внимательное отношение к молодым сотрудникам и за исключительную доброту и широту души. Автор благодарит Федорова Глеба Петровича за готовность обсуждать самые разные вопросы науки и философии, а также за хорошее чувство юмора и за дружеское отношение. Автор благодарит Шайхайдарова Раиса, Антонова Владимира и Терезу Хонигль-Декринис за гостеприимство и  заботу, а также за помощь в освоении азов нанофабрикации. Автор благодарит Коростылева Евгения, Киртаева Романа, Морозова Сергея и Негрова Дмитрия за огромную работу по поддержанию ЦКП МФТИ в рабочем состоянии, без чего невозможно было бы достичь результатов, представленных в диссертации. Автор благодарит аспиранта Кадырметова Шамиля, студентов Васенина Андрея, Гунина Сергея. за возможность передать им свой скромный опыт и за снисходительность к недостаткам автора как научного руководителя.  Автор благодарит весь коллектив лаборатории ИКС МФТИ, в частности Храпача И.Н., Болгара А.Н., Калачеву Д.С., Стрельникова А.С., Воробьеву С.Н, Кулакову А.И., Зотову Ю.В., Юрса В.Б., Сандуляну Ш.В., и крайне ценит возможность общения и совместной работы с каждым из своих коллег.  Автор благодарит сотрудников лаборатории сверхпроводящих метаматериалов МИСиС Беседина И.С., Абрамова Н.В., Чичкова В.В. и руководителя лаборатории проф. Устинова А.В. за гостеприимство, за готовность к коллаборации, за возможность одолжить необходимые микроволновые компоненты и всяческую поддержку любой полезно деятельности. Наконец, автор бесконечно благодарен своим родителям Юрию Викторовичу и Светлане Васильевне, любимой супруге Анне и своим детям Елизавете, Василисе, Федору и Вере за любовь, терпение, поддержку и понимание, без которых диссертация не была бы завершена. 

