\chapter{Элементы физики сверхпроводящих квантовых цепей} \label{c1}

В Главе \ref{c1} кратко излагаются элементы физики сверхпроводимости, СКЦ и квантовой оптики, необходимые для изложения и обсуждения результатов диссертационной работы. Вначале кратко описываются причины возникновения сверхпроводимости и изгалаются основные физические свойства сверхпроводника. Затем подробно рассматривается эффект Джозефсона в туннельном SIS-переходе, описываются основные модели описания эффекта Джозефсона приводятся основные свойства такого SIS-перехода (далее называемого джозефсоновским переходом) в различных режимах работы. Делается вывод о том, почему джозефсоновский переход находит широкое применение в классической и квантовой сверхпроводящей электронике. Описывается общий формализм квантования сверхпроводящей электрической цепи и рассчитываются параметры некоторых видов СКЦ --- потоковый кубит, трансмон, а также rf-SQUID. 
Отдельный раздел посвящен элементам квантовой оптики --- науки, которая описывает квантование электромагнитного поля и эффекты взаимодействия такого поля с атомами и молекулами. 
\section{Введение в физику сверхпроводимости} \label{s1_sc_phys}
Явление сверхпроводимости было открыто в 1911 г. в лаборатории Х. Камерлинг-Оннеса, и практически сразу началось интенсивное как теоретическое, так и экспериментальное изучение данного явления. Оказалось, что некоторые металлы при температурах $T<T_c$ демонстрируют целый ряд необычных физических свойств. Опуская подробности, перечислим наиболее интересные свойства сверхпроводников:
\begin{enumerate}
	\item Полное отсутствие сопротивления электрическому току;
	\item Полное вытеснение магнитного поля из объема сверхпроводника (эффект Мейсснера);
	\item Квантование магнитного потока через замкнутый контур из сверхпроводящего металла;
	\item Скачкообразное возрастание теплоемкости металла при переходе через $T_c$.
\end{enumerate}	
Объяснить поведение сверхпроводников на феноменологическом уровне удалось с помощью классической теории Лондонов, которая опирается на двухжидкостную модель электронов в металле: часть электронов предполагаются <<сверхпроводящими>>, т.е. способными переносить электрический ток в отсутствие внешнего электрического поля. Одного этого предположения практически достаточно для того, чтоб объяснить многие свойства сверхпроводников, в частности, идеальный диамагнетизм. Квантовое обобщение этой теории было построено Гинзбургом и Ландау на основе теории фазовых переходов II рода. Ключевым предположением теории ГЛ было введение общей волновой функции сверхпроводящих электронов $\psi(\vec{r}) = \sqrt{\Delta(\vec{r})} e^{i\varphi{\vec{r}}}$ в металле и её рассмотрение в качестве параметра порядка, характеризующего фазовый переход. Некоторые вопросы, на которые теория Лондонов давала качественно неправильный ответ (например, значение поверхностной энергии границы между нормальной и сверхпроводящей фазами), были верно описаны с помощью теории ГЛ. Строго говоря, эта теория применима для описания сверхпроводимости в случае $T_c-T \ll T_c$\footnote[1]{Имеется также ограничение применимости теории, связанное с тем, что при $T$ очень близком к $T_c$ становятся важными флуктуационные эффекты.}, но оказывается, что для многих практически важных задач решение на основе теории ГЛ качественно совпадает с выводами микроскопической теории сверхпроводимости.

Несмотря на значительный прогресс в объяснении многих эксперименальных свойств сверхпроводников, истинный механизм возникновения сверхпроводимости долгое время оставался неописанным. Одним из результатов, указавшим на причину сверхпроводимости, стал изотопический эффект: для различных изотопов сверхпроводника в эксперименте наблюдается соотношение $T_c \cdot M^{\alpha}=\text{const}$. Следовательно, сверхпроводимость возникает из-за взаимодействия электронов с кристаллической решеткой. При детальном теоретическом анализе было выявлено, что возможен процесс эффективного притяжения электронов друг к другу посредством обмена фононами. В свою очередь, Купер показал, что даже малое отрицательное (притягивающее) взаимодействие между электронами дает очень необычный результат. Состояние металла, в котором все электроны занимают состояния с $E<E_F$, даже при $T=0$ не является основным. В металле могут возникать определенного рода парные возбуждения, при которых два электрона с противоположными квазиимпульсами и спинами занимают состояния с энергией $E\approx E_F + \Delta$. Такие возбуждения называются \textit{куперовскими парами}, и согласно теории, построенной Бардиным, Купером и Шриффером, полная энергия состояния с некоторым количеством куперовских пар оказывается меньше по сравнению с энергией состояния металла без пар, на величину порядка $N(0)\Delta^2$. Параметр $\Delta$ определяется характером электрон-фононного взаимодействия. Его значение определяет критическую температуру: $\Delta = 1.76 k_b T_c$, среднее количество куперовских пар в металле: $\Delta N(0) \approx k_b T_c/E_F$, и кроме того, является средней энергей связи в паре в расчете на один электрон. По этой причине он называется \textit{энергетической щелью}: для разрыва куперовской пары необходимо затратить энергию $2\Delta$, при этом пара распадётся на два квазичастичных возбуждения. Волновую функцию основного состояния пар в БКШ-теории можно записать как \cite{Tinkham}:
\begin{equation}
\ket{\psi_\varphi} = \prod_{\vec{k}}^{}(|u_{\vec{k}}|+|\nu_{\vec{k}}|e^{i\varphi}c^\dag_{\vec{k},\uparrow}c^\dagger_{-\vec{k}, \downarrow})\ket{\phi_0}.
\end{equation}  
Чрезвычайно важным обстоятельством является наличие фазового множителя при амплитуде вероятности рождения куперовской пары $|\nu_{\vec{k}}|$. Фаза $\varphi$ одна и та же для каждой пары и иллюстрирует то факт, что спаренные электроны образуют единое квантовое состояние. Для объемного сверхпроводника эта фаза не зависит от координаты. Можно показать, что в состоянии $\ket{\psi_\varphi}$ полное число пар не определено, однако, относительная дисперсия стремится к нулю по мере увеличения среднего числа электронов в системе, поэтому такое приближение можно считать разумным. Из состояний с $\ket{\psi_\varphi}$ c различными фазами можно получить состояние с определённым числом куперовских пар $\ket{\psi_n}$. Для этого достаточно заметить, что слагаемые в состояние $\ket{\psi_\varphi}$, отвечающие числу пар $N$ имеют фазовый множитель $e^{iN{\varphi}}$, и при усреднении по фазам все остальные слагаемые дадут нулевой вклад:
\begin{equation}
\ket{\psi_n} = \int_{0}^{2 \pi}e^{-i n \varphi} \ket{\psi_\varphi}.
\end{equation} 
Соотношение между $\ket{\psi_\varphi}$ и $\ket{\psi_n}$ имеет такой же вид, как и для векторов состояния $\ket{x}$~и~$\ket{p}$ свободной частицы, то есть, $n$ и $\varphi$ в сверхпроводнике являются канонически сопряженными переменными. Для них можно вывести коммутационное соотношение $[\hat{n}, \hat{\varphi}]=-i$ и соотношение неопределенностей $\Delta\varphi \cdot \Delta n \approx \hbar$. Таким образом, изолированный остров сверхпроводника обладает макроскопической квантовой степенью свободы. 